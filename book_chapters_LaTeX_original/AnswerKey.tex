%\documentclass[a4paper,12pt]{book}
%\usepackage[utf8]{inputenc}
%\usepackage{graphicx}
%\usepackage{amsmath}
%\DeclareMathOperator\cis{cis}
%
%\begin{document}
%
%\author{TeXstudio Team}
%\title{Solutions Manual}
%\date{September 2015}
%
%\frontmatter
%\maketitle
%\tableofcontents
%
%\mainmatter
%\chapter{Foward}
%\chapter{In the beginning...}
%\chapter{Complex Numbers}

\chap{Answer Key}{AnswerKey}

\section{Solutions for``Complex Numbers''}
\noindent\textbf{\textit{ (Chapter \ref{complex})}}\bigskip

\noindent\textbf{Exercise \ref{exercise:complex:findk}:} \\
$z=3+i$\\
$z^{2}-6z+k=0$\\
$3^{2}+6i+i^{2}-6(3+i)+k=0$\\
$8-18+k=0$\\
$k=10$\\
\\
\textbf{Exercise \ref{exercise:complex:tableentries}:}\\
The last column of the table is:\\
Additive identity: $(a+bi)+0=0+(a+bi)=a+bi$\\
Additive inverse : $(a+bi)+(-a-bi)=(-a-bi)+(a+bi)=0$\\
Associative law  : $(a+bi)+[(c+di)+(e+fi)]=[(a+bi)+(c+di)]+(e+fi)$\\
Commutative law: $(a+bi)+(c+di)=(c+di)+(a+bi)$\\
\\
\textbf{Exercise \ref{exercise:complex:14}:}\\
Real number 0 does not have multiplicative inverse because any number times 0 will be 0.\\
\\
\textbf{Exercise \ref{exercise:complex:15}:}\\
Multiplicative identity: $(a+bi)(1+0i)=(1+0i)(a+bi)=(a+bi)$\\
Multiplicative inverse: $$(a+bi)\frac{a-bi}{a^{2}+b^{2}}= \frac{a-bi}{a^{2}+b^{2}}(a+bi)=1$$\\
Show similar examples for associative law and commutative law.\\
\\
\textbf{Exercise \ref{exercise:complex:16}:}\\
Use FLOI for $[(a+bi)(c+di)](e+fi)$ and $(a+bi)[(c+di)(e+fi)]$ to show that they are equal.\\
\\
\textbf{Exercise \ref{exercise:complex:18}:}\\
b. $ (4-5i)-(\overline{4i-4})=8-i$\\
\\
c. $(9-i)(\overline{9-i})=82$\\
\\
f. $(\overline{\sqrt{3}-i})^{-1}=\displaystyle\frac{\sqrt{3}-i}{10}$\\
\\
g. $\overline{(\sqrt{3}-i)^{-1}}=\displaystyle\frac{\sqrt{3}-i}{10}$\newline
\\
h. $(\overline{(\overline{4-9i})^{-1}})^{-1}=4-9i$\\
\\
i. $(a+bi)(\overline{a+bi})=a^{2}+b^{2}$\\
\\
j. $(a+bi)+(\overline{a+bi})=2a$\\
\\
\textbf{Exercise \ref{exercise:complex:cxprops}:}\\
Let $z=a+bi$ and $w=c+di$ for the following problems:\\
a. $(\bar{z})(\bar{w})=(a-bi)(c-di)=...=(\overline{zw})$\\
\\
d. $z.\bar{z}=(a+bi)(a-bi)=a^{2}-b^{2}i^{2}=...=\left|z\right|^{2}$\\
\\
e. $\left|zw\right|=\left|(a+bi)(c+di)\right|=...=\left|ac-bd+(ad+bc)i\right|= \ldots \\
=\sqrt{(a^{2}+b^{2})(c^{2}+d^{2})}=\left|z\right|\left|w\right|$\\
\\
f. $\left|z^{3}\right|=\left|z\cdot z \cdot z\right|=...=\left|z\right|\cdot\left|z\right|\cdot\left|z\right|=\left|z\right|^{3}$, and use result from problem (e) above.\\
\\
h. $\left|z^{-1}\right|=\left|\displaystyle\frac{a-bi}{a^{2}+b^{2}}\right|=...=\displaystyle\frac{1}{\left|z\right|}$\\
\\
i. $(\bar{z})^{-1}=(a-bi)^{-1}=...=\overline{(z^{-1})}$\\
\\
\textbf{Exercise \ref{exercise:complex:abs1}:}\\
Let $z=a+bi$ for the following calculations:\\
a. If $z$ is a pure real number $\rightarrow b=0 \rightarrow \bar{z}=a-0i=a=z$.\\
If $\bar{z}=z \rightarrow a-bi=a+bi \rightarrow ... \rightarrow b=0 \rightarrow $ z is a pure real number.\\
\\
b. If $z$ is pure imaginary $\rightarrow a=0 \rightarrow ... \rightarrow \bar{z}=-z$.\\
If $\bar{z}=-z \rightarrow (a-bi)=-(a+bi) \rightarrow ... \rightarrow a=0 \rightarrow $ z is pure imaginary.\\
\\
\textbf{Exercise \ref{exercise:complex:21}:}\\
\\
a. $2\cis(\displaystyle\frac{\pi}{6})=2\cos\frac{\pi}{6}+2i\sin\frac{\pi}{6}=...=\sqrt{3}+i$\\
\\
e. $\displaystyle\frac{\sqrt{2}}{2}-\frac{\sqrt{6}}{2}i$\\
\\
f. $-\displaystyle\frac{\sqrt{21}}{14}+\frac{\sqrt{7}}{14}i$\\
\\
\textbf{Exercise \ref{exercise:complex:22}:}\\
\\
f. $2\cis\displaystyle\frac{\pi}{6}$\\
\\
i. $2\sqrt{3}\cis\displaystyle\frac{7\pi}{4}$\\
\\
k. $10\cis\displaystyle\frac{5\pi}{4}$\\
\\
\textbf{Exercise \ref{exercise:complex:23}:}\\
a. It's a disk (interior of the circle centered at the origin with radius 2).\\
\\
c. It's a circle centered at (0,1) with radius 2.\\
\\
d. It's a circle centered at (3,0) with radius 3.\\
\\
\textbf{Exercise \ref{exercise:complex:24}:}
\begin{align*}
z.w & = (r\cis\theta).(s\cis\phi)\\
&= r(\cos\theta+i\sin\theta).s(\cos\phi+i\sin\phi)\\
&= rs(\cos\theta+i\sin\theta)(\cos\phi+i\sin\phi)\\
&= ...\\
&= rs[\cos(\theta+\phi)+i\sin(\theta+\phi)]\\
&= rs\cis(\theta+\phi)
\end{align*}
\textbf{Exercise \ref{exercise:complex:polar_z_inv}:}\\
\\
a. $w=z^{-1}=...=\displaystyle\frac{1}{13}\cis\frac{9\pi}{7}$
Sum of the argument of $z$ and $w$ is $2\pi$.\\
\\
c. Let $z=r\cis\theta$ and $w=s\cis\phi$\\
$z^{-1}=...=\displaystyle\frac{1}{r}\cis(-\theta)=\frac{1}{r}\cis(2\pi-\theta)$\\
So if $w=z^{-1}$ then $s=\displaystyle\frac{1}{r}$ and $\phi=2\pi-\theta$.\\
Therefore, if $s=\displaystyle\frac{1}{r}$ and $\phi=2\pi-\theta$ we will have $z\cdot w=1$.\\
\\
\textbf{Exercise \ref{exercise:complex:25}:}\\
\\
b. $...=14\cdot \displaystyle\frac{1}{7}\cis\frac{10\pi}{5}=2$\\
\\
d. $...=42\sqrt{2}\cis\displaystyle\frac{13\pi}{60}$\\
\\
\textbf{Exercise \ref{exercise:complex:26}:}\\
\\
a. $...=\displaystyle\frac{5}{4}+\frac{5\sqrt{3}}{4}i$\\
\\
c. $...=8\cis\displaystyle\frac{\pi}{12}$\\
\\
\textbf{Exercise \ref{exercise:complex:27}:}
\begin{align*}
[r\cis\theta]^{2} & = (r\cis\theta)(r\cis\theta)\\
& = rr\cis(\theta+\theta)\\
& = r^{2}\cis(2\theta)\\
\end{align*}
\textbf{Exercise \ref{exercise:complex:28}:}\\
Use the proposition and result from the previous Exercise, we'll have $[r\cis\theta]^{3}=r^{3}\cis(3\theta)$.\\
\\
\textbf{Exercise \ref{exercise:complex:31}:}\\
\\
c. $...=2^{5}\cis\displaystyle\frac{5\pi}{6}=...=-16\sqrt{3}+16i$\\
\\
e. $...=\displaystyle\frac{4\cis(7\pi)}{16}=-\frac{1}{4}$\\
\\
f. $...=2^{12}\cis(15\pi)=-2^{12}$\\
\\
\textbf{Exercise \ref{exercise:complex:cos form}:}\\
a. Use Moivre's Theorem: $[r\cis\theta]^{n}=r^{n}\cis(n\theta)$\\
$z^{3}=[r\cis\theta]^{3}=r^{3}\cis(3\theta)$. Therefore:
\begin{align*}
\cos(3\theta)+i\sin(3\theta) & =[\cos\theta+i\sin\theta]^{3}\\
& = ...\\
& = \cos^{3}\theta-3\sin^{2}\theta\cos\theta+i(3\cos^{2}\theta\sin\theta-\sin^{3}\theta)\\
& = \cos\theta(\cos^{2}\theta-3\sin^{2}\theta)+i\sin\theta(3\cos^{2}\theta-\sin^{2}\theta)\\
\end{align*}
Apply real part = real part and imaginary part = imaginary part, we have:\\
$\cos(3\theta)=\cos\theta(\cos^{2}\theta-3\sin^{2}\theta)$\\
$\sin(3\theta)=\sin\theta(3\cos^{2}\theta-\sin^{2}\theta)$\\
\\
b. Use result from (a) to prove $\cos(3\theta)=...=4\cos^{3}\theta-3\cos\theta $.\\
\\
\textbf{Exercise \ref{exercise:complex:cos form2}:}\\
b. Polar representation of $\bar{z}$ is $r\cis(-\theta)$ or $r\cis(2\pi-\theta)$.\\

\section{Solutions for ``Modular Arithmetic''}
\noindent \textbf{\textit{ (Chapter \ref{modular})}}\bigskip

\noindent\textbf{Exercise \ref{exercise:modular:racetrack_displacements}:}\\
a. Net displacement $=346-432+99=13$.\\
b. Net displacement $=44+13.53=733$.\\
\\
\textbf{Exercise \ref{exercise:modular:racetrack_positions}:}\\
a. Position: mod(13,5)=3.\\
   Position: mod(733,5)=3.\\
b. Both cases have the same position.\\
\\
\textbf{Exercise \ref{exercise:modular:equivdef}:}\\
by m, of b, by m, $r>s$, $a=pm+r$, $b=qm+s$, $a-b=(p-q)m+(r-s)$, between 0 and m, $r-s>0$, $r-s<m$, between 0 and m, by m.\\
\\
\textbf{Exercise \ref{exercise:modular:eqproof}:}\\
$a \equiv b$ (mod n) $\rightarrow n|(a-b)$ (Prop.\ref{proposition:modular:equivalence_alt})\\
$b \equiv c$ (mod n) $\rightarrow n|(b-c)$\\
$\rightarrow a-b=kn$ and $ b-c=ln$ (k,l are integers).\\
$\rightarrow a-c=(k+l)n \rightarrow n|(a-c) \implies a \equiv c$ (mod n).\\
\\
\textbf{Exercise \ref{exercise:modular:jan25}:}\\
a. 3 mod 7 = 3 and 25 mod 7 = 4 $\implies$  Jan 3 is not Thursday.\\
b. 7 doesn't divide $(31 - 25) \implies 31 \not\equiv 25$ (mod 7). So Jan 31 is not a Thursday.\\
\\
\textbf{Exercise \ref{exercise:modular:22}:}\\
c. $101 \equiv 29$ (mod 6): True\\
e. True\\
f. False. Can change it to $1476532 \equiv -71832772$ (mod 10)\\
\\
\textbf{Exercise \ref{exercise:modular:27}:}\\
$b \le n$, $-b \ge -n$, $a-b \ge -n$, $a \le n$ and $b \ge 0$, $-b \le 0$, $a-b \le n$\\
a - b is between -n and n, a-b is a multiple of n, n between -n and n is 0, a - b = 0, a = b\\
\\
\textbf{Exercise \ref{exercise:modular:28}:}\\
Example 3: Thursday (total 113 days from Dec 24, 2011 to Apr 15, 2012 and 113 mod 7 = 1).\\
Example 4: Feb 14, 2010 is Chinese New Year day. Feb 14, 2011 is 365 days after that.\\
365 mod 354 = 11 so Chinese New Year in 2011 is Feb 3.\\
Chinese New Year in 2012 is Jan 23 (354 days after Feb 3).\\
In 2009, Chinese New Year is Feb 25.\\
\\
\textbf{Exercise \ref{exercise:modular:UPCSymbols}:}\\
a. It's a valid UPC because $30 \equiv 0$ (mod 10).\\
b. Total sum is $33 \not\equiv 0$ (mod 10).\\
d. Transposition will be detected because the sum is $24 \not\equiv 0$ (mod 10).\\
Transposition error that can't be detected is 005000300426.\\
e. Error is detectable.\\
f. Suppose one code has the following digits: $d_1,d_2,d_3,...,d_{10}$ and the other code has the following digits: $e_1,e_2,e_3,...,e_{10}$ are both valid and suppose that the corresponding digits at each position are equal except for the $n^{th}$ digit $\implies d_n \neq e_n$ with $1 \le n \le 10$.\\
If n is even:\\
$(sum + d_n) \equiv (sum + e_n)$ (mod 10)\\
$\implies d_n \equiv e_n$ (mod 10) because $d_n, e_n \le 9$\\
$\implies d_n = e_n$\\
It contradicts to the original supposition.\\
If n is odd:\\
$(sum+3d_n) \equiv (sum+3e_n)$ (mod 10)\\
$\implies 10|(sum+3d_n-sum-3e_n)$\\
$\implies 10|3(d_n-e_n)$\\
Since $d_n-e_n \le 10$, the only solution to make 10 divides $3(d_n-e_n)$ is $3(d_n-e_n)=0 \implies d_n=e_n$.\\
It also contradicts to the original supposition.\\
Therefore UPC error detection scheme detects all single digit errors.\\
\\
\textbf{Exercise \ref{exercise:modular:mod_eq_1}:}\\
b. $25+x \equiv 6 \pmod{12}$\\
$1+x \equiv 6 \pmod{12}$\\
$x= 12k+6-1$\\
$x=5+12k$\\
\\
\textbf{Exercise \ref{exercise:modular:mod_eq_2}:}\\
g. $x=-1+7k$\\
i. $5x+1 \equiv 13 \pmod{26}$\\
$x= \displaystyle\frac{26k+12}{5}$\\
Let $k=3+5n$\\
$x=18+6n$\\
h. $5x+1 \equiv 13 \pmod{23}$\\
$x= \displaystyle\frac{23k+12}{5}$\\
Let $k=1+5n$\\
$x=7+23n$\\
\\
\textbf{Exercise \ref{exercise:modular:44}:}\\
a. Since $a \equiv b \pmod{n}$ and $c \equiv d \pmod{n}$\\
then $a=b+sn$ and $c=d+tn$\\
$\implies a.c =(b+sn)(d+tn)=bd+n(bt+sd+stn)$\\
By definition of $\odot$: $b \odot d \pmod{n} \to$ there is some integer q such that $bd=(b \odot d) +qn$\\
So we have: $ac=(b \odot d)+qn+n(bt+sd+stn)=b \odot d + n(q+bt+sd+stn)$\\
$\implies ac \equiv b \odot d \pmod{n}$\\
b. $x \ominus y=r iff x-y=r+sn$ and $r \in Z_n$\\
c. We have $a=b+sn$ and $c=d+tn$\\
$\implies a-c=b+sn-d-tn=b-d+n(s-t)$\\
By definition of $\ominus$: $b \ominus d \pmod{n} \to$ there is some integer k such that $b-d=b \ominus d +kn$\\
Therefore $a-c=b \ominus d +n(k+s-t) \implies a-c=b \ominus d \pmod{n}$\\
\\
\textbf{Exercise \ref{exercise:modular:ops}:}\\
a. $a+c \equiv b \oplus d \pmod{n}$ (Prop. 43)\\
Let call $a+c=k$ and $b \oplus d=r$\\
$\implies (a+c)+e=k+e$\\
$\implies (a+c)+e \equiv r \oplus f \pmod{n}$ (Prop. 43)\\
$\implies (a+c)+e \equiv (b \oplus d) \oplus f \pmod{n}$\\
b. $ac \equiv b \odot d \pmod{n}$ (Prop. 43)\\
Let $ac=k$ and $b \odot d =r$\\
$\implies k \equiv r \pmod{n}$\\
$\implies k+e \equiv r \oplus f \pmod{n}$ (Prop. 43)\\
$\implies ac+e \equiv (b \odot d) \oplus f \pmod{n}$\\
c. $a+c \equiv b \oplus d \pmod{n}$ (Prop. 43)\\
Let $a+c=k$ and $b \oplus d =r$\\
$k \equiv r \pmod{n}$\\
$ke \equiv r \odot f \pmod{n}$ (Prop. 43)\\
$\implies (a+c)e \equiv (b \oplus d) \odot f \pmod{n}$\\
\\
\textbf{Exercise \ref{exercise:modular:49}:}\\
a. $\equiv 1 \pmod{5}$\\
b. $\equiv 2 \pmod{5}$\\
c. $\equiv 1 \pmod{5}$\\
d. $\equiv 0 \pmod{5}$\\
\\
\textbf{Exercise \ref{exercise:modular:52}:}\\
Suppose $a,b \in Z_n$ then by the definition of modular multiplication (Def. 42), we have $a \odot b =r \pmod{n}$ where $r \in \{0,1,...,n-1\}$\\
$\implies a \odot b \in Z_n$\\
$\implies Z_n$ is closed under modular multiplication.\\
\\
\textbf{Exercise \ref{exercise:modular:53}:}\\
b. Rational numbers: closed under addition, subtraction, multiplication.\\
d. Positive rational numbers: closed under addition and multiplication.\\
\\
\textbf{Exercise \ref{exercise:modular:54}:}\\
Let $z_1=a+bi$ and $z_2=c+di$, then:\\
$z_1z_2=(a+bi)(c+di)=(ac-bd)+(ad+bc)i$ is a complex number\\
$\implies$  Complex numbers are closed under multiplication.\\
\\
\textbf{Exercise \ref{exercise:modular:56}:}\\
Given any a $\in Z_n, (n>1)$ then $a \odot 1$ is computed as follows:\\
a. Compute $a.1$ using ordinary multiplication\\
b. taking the remainder mod n \\
Since $a1=a$, and $0 \le a <n$, it follows that the remainder is also a.\\
Hence $a \odot 1=a$.\\
Similarly, we can show that $1 \odot a=a$.\\
So, 1 satisfies the definition of multiplicative identity for $Z_n$ when $n>1$.\\
When $n=1$, $Z_1=\{0\} \to$ multiplicative identity for $Z_1$ is 0.\\
\\
\textbf{Exercise \ref{exercise:modular:58}:}\\
a. $0 \in Z_n$ and $0 \oplus 0=0 \to$ additive inverse of $0 \in Z_n$ is 0.\\
b. Suppose $a \in Z_n \setminus \{0\}$ and let $a'=n-a$\\
We have $0<a<n \implies n>a'>0 \implies a' \in Z_n$\\
We also have $a \oplus a'=a+n-a \pmod{n}=n \pmod{n}=0 \pmod{n}$\\
Similarly, we have $a' \oplus a=0 \pmod{n}$\\
$\implies a'$ is the additive inverse of a.\\
\\
\textbf{Exercise \ref{exercise:modular:60}:}\\
a. n = 3\\
b. n = 7\\
\\
\textbf{Exercise \ref{exercise:modular:64}:}\\
a. 0 has no multiplicative inverse under $\odot \implies Z_n$ is not a group.\\
b. $Z_3 \setminus \{0\}$ is a group under $\odot$.\\
\\
\textbf{Exercise \ref{exercise:modular:71}:}\\
a. We have $r_2=b-r_1q_2$ (Prop. 68)\\
We also have $r_1=a-bq_1$ (from previous step)\\
$\implies r_2=b-(a-bq_1)q_2=b(1+q_1q_2)-aq_2$ where $q_1$ and $q_2$ are integers.\\
$\implies r_2$ can also be written in the form $r_2=na+mb$ where n and m are integers.\\
b. For $k>2$, we have:\\
$r_{k-2}=na+mb$\\
$r_{k-1}=sa+tb$\\
We also have $r_k=r_{k-2}-r_{k-1}q_k$ (Prop. 68)\\
$\implies r_k=na+mb-(sa+tb)q_k=a(n-sq_k)+b(m-tq_k)$\\
$\implies r_k$ can also be written in the form $r_k=va+wb$ where $v$ and $w$ are integers.\\
c. From Prop. 68, $r_k$ is the gcd of two numbers a and b.\\
From b. we know that $r_k$ can be written in the form $na+mb$.\\
$\implies$  the gcd of two numbers a and b can be written in the form $na+mb$ where $n$ and $m$ are integers.\\
\\
\textbf{Exercise \ref{exercise:modular:90}:}\\
a. $Z_5 \setminus \{0\}$ forms a group.\\
b. $Z_7 \setminus \{0\}$ forms a group.\\
c. $Z_9 \setminus \{0\}$ doesn't form a group because element 3 doesn't have multiplicative inverse.\\
d. $Z_n \setminus \{0\}$ with n is a prime number will form a group under $\odot$.\\
\\
\textbf{Exercise \ref{exercise:modular:93}:}\\
Let $a \in Z_n \setminus \{0\}$, we need to find $x$ is the multiplicative inverse of $a$\\
$\implies ax \equiv 1 \pmod{n}$.
According to Prop. 78, we can only find $x$ if 1 is an integer multiple of the gcd of $a$ and $n$\\
$\implies$  1 is the gcd of $a$ and $n$ in order to find $x$, which is the multiplicative inverse of $a$.\\
$\implies a$ is relatively prime to $n$ to make $x$ exist.\\
If $a$ is not relatively prime to $n \to$ they have a gcd which is bigger than 1 $\implies$  1 is not an integer multiple of the gcd of $a$ and $n \to$ we can't find $x$ (Prop. 78) $\implies a$ doesn't have an inverse under multiplication (mod n).\\
\\
\textbf{Exercise \ref{exercise:modular:94}:}\\
For $Z_n \setminus \{0\}$, if n is not prime, there is $1<a<n \in Z_n$ that $a|n$\\
$\implies$  the gcd of $a$ and $n$ is bigger than 1 $\implies a$ does not have a multiplicative inverse (according to Exercise 86) $\implies Z_n \setminus \{0\}$ is not a group under $\odot$.\\


\section{Solutions for ``Set Theory''}
\noindent\textbf{\textit{ (Chapter \ref{sets})}}\bigskip

\textbf{Exercise \ref{exercise:sets:7}:}\\
b. 4 elements\\
\\
\textbf{Exercise \ref{exercise:sets:12}:}\\
b. Let $A_1=\{ 1+i,2+2i\}$, $A_2=\{ 3+3i,4+4i\}$, and $A_3=\{5+5i,6+6i\}$ then we have:
$A_1,A_2,A_3 \subset C$, $A_1 \cap A_2=\emptyset$, $A_1 \cap A_3=\emptyset$, $A_2 \cap A_3=\emptyset$, and $A_1 \cap A_2 \cap A_3=\emptyset$.\\
\\
\textbf{Exercise \ref{exercise:sets:14}:}\\
b. $\bigcap^{n}_{i=1} \{ 1,2,...,i \}=\{1\}$\\
\\
c. $\bigcap^{\infty}_{i=1} \{ 1,2,...,i \}=\{1\}$\\
\\
\textbf{Exercise \ref{exercise:sets:15}:}\\
a. $A_i=\{x\in R|i-1 \le x\le i\}$ with $i=1,2,3,...$\\
b. $A_i=\{x\in R|i-1 < x\le i\}$ with $i=1,2,3,...$\\
\\
\textbf{Exercise \ref{exercise:sets:16}:}\\
a. $A_1=\{...,-3,-2,-1\} $\\
$A_2=\{0\}$\\
$A_3=\{1,2,3\}$\\
$A_4=\{4,5,6,7,...\}$\\
b. $A_1=\{x\in R|x<0\}$\\
$A_2=\{0\}$\\
$A_3=\{x\in R|0<x<1\}$\\
$A_4=\{x\in R|x\ge 1\}$\\
\\
\textbf{Exercise \ref{exercise:sets:18}:}\\
$A\subset B$. The largest subset of $B$ that is disjoint from $A$ is $B\setminus A$\\
\\
\textbf{Exercise \ref{exercise:sets:20}:}\\
a. $(A\cap B)\setminus C=\emptyset$\\
b. $A\cap B\cap C\cap D=\emptyset$\\
\\
\textbf{Exercise \ref{exercise:sets:21}:}\\
a. $A\cap B=\{2\}$\\
c. $A'\cap B'=\{x\in N|$ x is odd and x is not prime$\}$\\
e. $(A\cup B)'=\{x\in N|$ x is odd and x is not prime$\}$\\
f. $A'\cup B'=\{x\in N|$ x is odd or x is not prime$\}$\\
\\
\textbf{Exercise \ref{exercise:sets:23}:}\\
First, suppose that $x$ is an element of $A$, then we have:\\
$x\in A$ (supposition) $\implies x\in A$ or $x\in B$ (logic) $\implies x\in A\cup B$ (def. of union)\\
Since every element of A is an element of $A\cup B$, it follows by the definition of subset that $A\subset (A\cup B)$\\
\\
\textbf{Exercise \ref{exercise:sets:26}:}\\
(Part 6 of Prop. 24)\\
a. Suppose $x$ is an element of $A\cup B$\\
$\implies x\in A\cup B$ (supposition)\\
$\implies x\in A$ or $x\in B$ (def. of union)\\
$\implies x\in B$ or $x\in A$ (logic)\\
$\implies x\in B\cup A$ (def. of union)\\
Since every element of $A\cup B$ is an element of $B\cup A$, we have $(A\cup B)\subset(B\cup A)$ (1)\\
Similarly, we can prove every element of $B\cup A$ is an element of $A\cup B$, so we have $(B\cup A)\subset (A\cup B)$ (2)\\
From (1) and (2) $\implies A\cup B = B\cup A$\\
b. Similarly, steps by steps as in part (a), we can prove that $A\cap B=B\cap A$\\
\\
\textbf{Exercise \ref{exercise:sets:30}:}\\
a.\begin{align*}
(A\cap B) \setminus B & =(A\cap B)\cap B'\\
& = A\cap (B\cap B')\\
& = A\cap \emptyset\\
& = \emptyset\\
\end{align*}
b.\begin{align*}
(A\cup B) \setminus B & =(A\cup B)\cap B'\\
& = B'\cap (A\cup B)\\
& = (B'\cap A)\cup(B'\cap B)\\
& = (B'\cap A)\cup \emptyset\\
& = B'\cap A\\
& = A\cap B'\\
& = A\setminus B\\
\end{align*}
c.\begin{align*}
A\setminus (B\cup C) & =A\cap (B\cup C)'\\
& = A\cap (B'\cap C')\\
& = A\cap A\cap(B'\cap C')\\
& = (A\cap B')\cap (A\cap C')\\
& = (A\setminus B)\cap (A\setminus C)\\
\end{align*}
\textbf{Exercise \ref{exercise:sets:31}:}\\
a. $S=\{a,b,c\}$\\
Subsets of S: $\{a\};\{b\};\{c\};\{a,b\};\{b,c\};\{c,a\};\{a,b,c\};\emptyset$\\
There are 8 subsets.\\
b. $S=\{a,b\}$. Similarly, there are 4 subsets of S.\\
c. $S=\{a,b,c,d\}$. There are 16 subsets of S as follows:\\
$\{a\};\{b\};\{c\};\{d\}$\\
$\{a,b\};\{a,c\};\{a,d\};\{b,c\};\{b,d\};\{c,d\}$\\
$\{a,b,c\};\{a,b,d\};\{b,c,d\};\{a,c,d\}$\\
$\{a,b,c,d\};\emptyset$\\
\\
\textbf{Exercise \ref{exercise:sets:cup_group}:}\\
$G$ is the set of subsets of the set $\{a,b,c\}$\\
a. $G$ is closed under the $\cup$ operation. Union of any element in $G$ will be still in $G$.\\
b. The identity is $\emptyset$. We have $A\cup \emptyset=A$.\\
c. The operation $\cup$ is associative.\\
d. The operation $\cup$ is commutative.\\
e. No element of $G$ has a unique inverse under the operation $\cup$.\\
For example: $\{a\}$ has no inverse $I$ to make $\{a\}\cup I=\emptyset$\\
f. So $G$ is not a group under the operation $\cup$ because elements in $G$ have no inverse.\\

\section{Solutions for ``Functions''}
\noindent\textbf{\textit{ (Chapter \ref{functions}})}\bigskip

\textbf{Exercise \ref{exercise:functions:ExOrderPairs}:}\\
$A=\{a,b,c,d\}$, $B=\{1,3,5,7,9\}$\\
b. function from $A \implies B$\\
d. function from $A \implies B$\\
f. function from $A \implies B$\\
h. not a function from $A \implies B$\\
\\
\textbf{Exercise \ref{exercise:functions:23}:}\\
a. Domain = $\{2,4,6,8,10\}$\\
b. Range = $\{7,9,11,13,15\}$\\
c. $g(6)=11$\\
d. $g(7)$ is undefined\\
e. $g=\{(2,7),(4,9),(6,11),(8,13),(10,15)\}$\\
g. $g(n)=n+5$\\
\\
\textbf{Exercise \ref{exercise:functions:27}:}\\
b. $A=\{a,b\}$, $B=\{c,d\}$\\
$f=\{(a,c),(b,d)\}$ Range: $\{c,d\}$\\
$g=\{(a,d),(b,c)\}$ Range: $\{d,c\}$\\
$h=\{(a,c),(b,c)\}$ Range: $\{c,c\}$\\
$k=\{(a,d),(b,d)\}$ Range: $\{d,d\}$\\
\\
\textbf{Exercise \ref{exercise:functions:11Exers}:}\\
c. $h(x)=x^2$\\
$h(x)$ is not one-to-one because $f(-1)=f(1)=1$ while $-1\neq 1$\\
d. $i(x)=3x+2$\\
Let $x_1,x_2 \in R$. If $f(x_1)=f(x_2)$, then we have:
$3x_1+2=3x_2+2 \implies x_1=x_2$\\
$\implies i(x)$ is a one-to-one function.\\
\\
%\textbf{Exercise 37:}\\
%a. $f$ is one-to-one because if $x_1\neq x_2$ then we have $f(x_1)\neq f(x_2)$ (contrapositive).\\
%b. $g$ is not one-to-one because $g(2)=g(3)=d$ while $2\neq 3$.\\
%\\
\textbf{Exercise \ref{exercise:functions:40}:}\\
e. $g: Z_8 \implies Z_8$ with $g(x)=x\odot 2$\\
$g(x)$ is not one-to-one because we have $g(1)=g(5)=2 \pmod{8}$ while $1\neq 5$.\\
f. $g: Z_7 \implies Z_7$ with $g(x)=x\odot 2$\\
$g(x)$ is one-to-one function because if $x_1\neq x_2$, then we will have $g(x_1)\neq g(x_2)$.\\
\\
\textbf{Exercise \ref{exercise:functions:45}:}\\
If the function $f$ is onto then the range of $f$ equals the codomain of $f$. There is no element in the codomain that is not in the range.\\
\\
\textbf{Exercise \ref{exercise:functions:OntoExers}:}\\
c. $c(x)=x^2$\\
$c(x)$ is not onto because there is a number $-2 \in R$ in the codomain that doesn't map to any element in the domain.\\
d. $d(x)=3x+2$\\
$d(x)$ is onto because for any element $m$ in the codomain, we can find\\ $x=\displaystyle\frac{m-2}{3}$ in the domain that maps to $m$.\\
\\
\textbf{Exercise \ref{exercise:functions:OntoExers-pairs}:}\\
a. It's onto. For any element in the codomain, there is an element in the domain thap maps to it.\\
b. It's not onto. Element $\diamondsuit$ doesn't map to any element in the domain.\\
\\
\textbf{Exercise \ref{exercise:functions:BijectRtoRExer}:}\\
d. $d(x)=-15x-12$\\
$d(x)$ is one-to-one because if $d(x_1)=d(x_2)$ then $x_1=x_2$.\\
$d(x)$ is onto because for any element y in the codomain, we have $x=\displaystyle\frac{y+12}{-15}$ in the domain that maps to $y$.\\
$\implies d(x)$ is a bijection.\\
e. $e(x)=x^3$\\
Similarly, we have $e(x)$ is both one-to-one and onto $\implies e(x)$ is a bijection.\\
\\
%\textbf{Exercise 62:}\\
%$f: R\implies R$ by $f(x)=ax+b$ with $a,b \in R$\\
%a. When $a\neq 0$: $f(x)$ is a bijection.\\
%b. When $a=0$: $f(x)$ is not a bijection.\\
%\\
\textbf{Exercise \ref{exercise:functions:RealWorldCompositionExer}:}\\
b. husband $\circ$ mother : father\\
e. mother $\circ$ sister : mother\\
f. daughter $\circ$ sister : niece\\
\\
\textbf{Exercise \ref{exercise:functions:func_comp_assoc}:}\\
$(h\circ(g\circ f))(x)=h(g(y))=h(w)=z$\\
$((h\circ g)\circ f)(x)=...=z$\\
$\implies (h\circ(g\circ f))(x)=((h\circ g)\circ f)(x)$\\
\\
\textbf{Exercise \ref{exercise:functions:ComposeExers-form}:}\\
a. $f(x)=3x+1$ and $g(x)=x^2+2$\\
$(f\circ g)(x)=3x^2+7$\\
$(g\circ f)(x)=9x^2+6x+3$\\
b. $f(x)=3x+1$ and $g(x)=\displaystyle\frac{x-1}{3}$\\
$(f\circ g)(x)=x$\\
$(g\circ f)(x)=x$\\
c. $f(x)=ax+b$ and $g(x)=cx+d$\\
$(f\circ g)(x)=acx+ad+b$\\
$(g\circ f)(x)=cax+cb+d$\\
\\
\textbf{Exercise \ref{exercise:functions:ComposeExers-pairs}:}\\
a. $g\circ f=\{(1,\clubsuit),(2,\diamondsuit),(3,\heartsuit),(4,\spadesuit)\}$\\
b. $g\circ f=\{(1,\clubsuit),(2,\clubsuit),(3,\clubsuit),(4,\clubsuit)\}$\\
\\
\textbf{Exercise \ref{exercise:functions:BijectionComposeExer}:}\\
$f: A\implies B$ and $g: B\implies C$\\
a. $f$ and $g$ are bijection $\implies$  $f$ and $g$ are one-to-one $\implies g\circ f$ is one-to-one (previous Exercise).\\
$f$ and $g$ are onto $\implies g\circ f$ is onto.\\
$\implies g\circ f$ is bijection.\\
\\
\textbf{Exercise \ref{exercise:functions:88}:}\\
Prove part (b) of previous Example\\
For any $y\in Y$, from the supposition, we know that $\exists x\in X$ such that $g(y)=x$ and for that particular $x$, we also have $f(x)=y$.\\
So $g(f(x))=g(y)=x$ for all $x\in X$\\
\\
\textbf{Exercise \ref{exercise:functions:VerifyInverseExers}:}\\
a. We have $f(g(x))=x+6-6=x$ and $g(f(x))=...=x$\\
$\implies g$ is an inverse of $f$.\\
c. $f(g(x))=g(f(x))=x \implies g$ is an inverse of $f$.\\
\\
\textbf{Exercise \ref{exercise:functions:92}:}\\
$g$ has an inverse.\\
\\
\textbf{Exercise \ref{exercise:functions:98}:}\\
a. $Id_A$ is one-to-one because if $Id_A(a_1)=Id_A(a_2)$ then $a_1=a_2$.\\
$Id_A$ is onto because for any $a\in$ codomain $A$, we can always have $a\in$ domain $A$ that maps to $a$.\\
$\implies Id_A$ is bijection $\implies Id_A$ is invertible (Prop. 89)\\
b. $Id_A^{-1}(a)=a$.\\
\\
\textbf{Exercise \ref{exercise:functions:InverseIdentityExers}:}\\
a. $f: A\implies B$ and $g: B\implies C$ are bijection\\
$\implies g\circ f$ is bijection, $f$ has an inverse, and $g$ has an inverse.\\
Let $a\in A$, $b\in B$, $c\in C$ and $f(a)=b$ and $g(b)=c$\\
$\implies (g\circ f)(a)=c$\\
$\implies (g\circ f)^{-1}(c)=a$\\
We also have $(f^{-1}\circ g^{-1})(c)=f^{-1}(b)=a$\\
$\implies (g\circ f)^{-1}=f^{-1}\circ g^{-1}$\\
b. Let $x\in X$ and $y\in Y$\\
If $g$ is the inverse of $f$, we will have $f(x)=y$ and $g(y)=x$\\
$\implies (f\circ g)(y)=f(g(y))=f(x)=y=Idy$ and $(g\circ f)(x)=g(y)=x=Idx$\\
On the other hand, if we have $(f\circ g)(y_1)=Idy_1=y_1$ where $g(y_1)=x_1 \implies f(x_1)=y_1$.\\
And if $(g\circ f)(x)=Idx \implies (g\circ f)(x_1)=...=x_1$\\
So we have $g$ is the inverse of $f$ if and only if $f\circ g=Idy$ and $g\circ f=Idx$\\
c. $f: X\implies Y$ is a bijection $\implies f$ is both one-to-one and onto\\
Let $x\in X$ and $y\in Y$ where $f(x)=y$\\
$\implies f$ has an inverse $f^{-1}(y)=x$\\
$[f^{-1}(y)]^{-1}=y=f(x)$\\
Therefore $(f^{-1})^{-1}=f$.\\

\section{Solutions for  ``Equivalence Relations''}
\noindent\textbf{\textit{ (Chapter \ref{EquivalenceRelationsChap})}}\bigskip

\noindent\textbf{Exercise \ref{exercise:EquivalenceRelationsChap:7}:}\\
c. Relations: $\{(a,1)\},\{(b,1)\},\{(a,1),(b,1)\},\emptyset$\\
d. $A$ x $A$=$\{(a,a),(a,b),(b,a),(b,b)\}$\\
Relations: $\{(a,a)\},\{(a,b)\}, \{(b,a)\}, \{(b,b)\}$\\
$\{(a,a),(a,b)\}, \{(a,a),(b,a)\}, \{(a,a),(b,b)\}$\\
$\{(a,b),(b,a)\}, \{(a,b),(b,b)\}, \{(b,a),(b,b)\}$\\
$\{(a,a),(a,b),(b,a)\}, \{(a,a),(a,b),(b,b)\}, \{(a,a),(b,a),(b,b)\}, \{(a,b),(b,a),(b,b)\}$\\
$\{(a,a),(a,b),(b,a),(b,b)\},\emptyset$\\
e. $A$ x $A$=$\{(a,a),(a,b),(a,c),(b,a),(b,b),(b,c),(c,a),(c,b),(c,c)\}$\\
There are $2^9=512$ binary relations on the set $A$\\
\\
\textbf{Exercise \ref{exercise:EquivalenceRelationsChap:RelationDef}:}\\
b. $R$ = $\{(a+bi,c+di)\in C$ x $C | a=c$ and $b=d$ where $a,b,c,d\in R\}$\\
d. $A=\{1,2,3\}$\\
\\
\textbf{Exercise \ref{exercise:EquivalenceRelationsChap:17}:}\\
b. By definition of subset: It's transitive and reflexive; it isn't symmetric.\\
\\
\textbf{Exercise \ref{exercise:EquivalenceRelationsChap:21}:}\\
a. If each element has an arrow starts from and come back to itself $\implies$  reflexive.\\
b. If all relation lines have arrow signs at both ends $\implies$  symmetric.\\
\\
\textbf{Exercise \ref{exercise:EquivalenceRelationsChap:EquivRelShowEx}:}\\
b. It' s an equivalence relation because it's reflexive, symmetric,and transitive.\\
c. Equivalence relation.\\
Prove that it's transitive:\\
Let $(a_1,b_1)\sim (a_2,b_2) \implies a_1b_2=a_2b_1$\\
    $(a_2,b_2)\sim (a_3,b_3) \implies a_2b_3=a_3b_2$\\
Then we have:\\
$a_1b_3$ = $a_1b_3\displaystyle\frac{b_2}{b_2}$ ($b_2\in N$ so $b_2\neq 0$)\\
= $a_1b_2\displaystyle\frac{b_3}{b_2}=a_2b_1\displaystyle\frac{b_3}{b_2}=a_3b_2\displaystyle\frac{b_1}{b_2}$\\
$\implies a_1b_3=a_3b_1$\\
$\implies (a_1,b_1)\sim (a_3,b_3$\\
\\
\textbf{Exercise \ref{exercise:EquivalenceRelationsChap:EquivClassEasyEx}:}\\
b. $[1]=\{1,3,5\}$\\
$[2]=\{2,4\}$\\
$[3]=\{1,3,5\}$\\
$[4]=\{2,4\}$\\
$[5]=\{1,3,5\}$\\
\\
\textbf{Exercise \ref{exercise:EquivalenceRelationsChap:EquivRelPropsPfEx}:}\\
a. Equivalence relation $\implies$  reflexive $\implies a \sim a \implies a\in[a]$\\
b. $a\in [a]$ (from part a) $\implies [a]\neq \emptyset$\\
c. Let $a_1\in \bigcup_{a\in S}[a] \implies a_1\in [a]$. We also have $[a]\subset S \implies a_1\in S$. So $\bigcup_{a\in S}[a]\subset S$.\\
Let $a_2 \in S$. From part (a), we have $a_2 \in [a_2] \implies a_2\in\bigcup_{a\in S}[a] \implies S\subset \bigcup_{a\in S}[a]$.\\
Therefore $\bigcup_{a\in S}[a]=S$.\\
d. Let $a_1,a_2 \in S$ such that $a_1 \sim a_2$. Choose $a_3\in[a_1] \implies a_3 \sim a_1$ (Def. 33)\\
$\implies a_3 \sim a_2$ (because $a_1 \sim a_2$, transitive property)\\
$\implies a_3 \in [a_2]$ (Def. 33)\\
$\implies [a_1]\subset[a_2]$ (1)\\
Choose $a_4\in[a_2]$ then, similarly, we have $a_4\in[a_1]$\\
$\implies [a_2]\subset[a_1]$ (2)\\
From (1) and (2), we have $[a_1]=[a_2]$.\\
e. Let $a_1,a_2 \in S$ such that $a_1 \not\sim a_2$\\
Suppose that $[a_1]\cap[a_2]\neq\emptyset \to$ there is an element $x\in[a_1]$ and $x\in[a_2]$.\\
$x\in[a_1] \implies x\sim [a_1] \implies [a_1]\sim x$\\
$x\in[a_2] \implies x\sim [a_2]$\\
$\implies [a_1]\sim[a_2]$ (transitive) $\implies$  it contradicts to the supposition that $a_1 \not\sim a_2$.\\
Therefore, $[a_1]\cap[a_2]=\emptyset$.\\
\\
\textbf{Exercise \ref{exercise:EquivalenceRelationsChap:DisjointEquivEx}:}\\
a. $a\in[a_1]\cap[a_2]$\\
b. $a\in[a_1]$ and $a\in[a_2]$\\
c. $a\sim a_1$ and $a\sim a_2$\\
d. part (d) tells us that $[a]=[a_1]$ and $[a]=[a_2]$\\
e. Therefore $[a_1]=[a]=[a_2]$\\
\\
\textbf{Exercise \ref{exercise:EquivalenceRelationsChap:ProveModEquiv}:}\\
$a\equiv b\pmod{n}$ iff $a-b=kn$ ($k\in Z$).\\
We have $a-a=0n \implies a\equiv a\pmod{n} \to$ reflexive.\\
(i) For symmetric:\\
$a\equiv b\pmod{n} \implies a-b=kn \implies b-a=-kn$\\
$-k\in Z$ because $-1,k\in Z$\\
$\implies b\equiv a\pmod{n}$\\
(ii) For transitive:\\
Let $a\equiv b\pmod{n}$ and $b\equiv c\pmod{n}$\\
$\implies a-b=kn$ and $b-c=ln$\\
$\implies a-c=(k+l)n$\\
$\implies a\equiv c\pmod{n}$\\
Therefore, equivalent mod n is an equivalence relation.\\
\\
\textbf{Exercise \ref{exercise:EquivalenceRelationsChap:47}:}\\
a. If r is the remainder when $a-b$ is divided by 3, then $\bar{a}-\bar{b}=\bar{r}$\\
b. If r is the remainder when $a.b$ is divided by 3, then $\bar{a}.\bar{b}=\bar{r}$\\
\\
\textbf{Exercise \ref{exercise:EquivalenceRelationsChap:54}:}\\
a. (i) $a_1\equiv a_2\pmod{n}$ and $b_1\equiv b_2\pmod{n}$\\
(ii) $a_1-a_2=k_1n$ and $b_1-b_2=k_2n$, where $k_1,k_2 \in Z$.\\
(iii) $(a_1+b_1)-(a_2+b_2)=(k_1+k_2)n$\\
(iv) $a_1+b_1 \equiv (a_2+b_2)\pmod{n}$\\
(v) that $\overline{a_1+b_1}=\overline{a_2+b_2}$\\
(vi) $\overline{a_1}+\overline{b_1}=\overline{a_2}+\overline{b_2}$\\
b. Suppose $\overline{a_1}=\overline{a_2}$ and $\overline{b_1}=\overline{b_2}$.\\
From the definition of equivalence class, it follows that $a_1\equiv a_2\pmod{n}$ and $b_1\equiv b_2\pmod{n}$.\\
By definition 42, it follows that $a_1-a_2=k_1n$ and $b_1-b_2=k_2n$ where $k_1,k_2 \in Z$.\\
By integer arithmetic, if follows that $(a_1-b_1)-(a_2-b_2)=(k_1-k_2)n$.\\
Since $k_1,k_2 \in Z$, it follows from def. 42 that $(a_1-b_1)\equiv (a_2-b_2)\pmod{n}$.\\
It follows from proposition 37 part (d) that $\overline{a_1-b_1}=\overline{a_2-b_2}$.\\
By definition 50 part (c), then $\overline{a_1}-\overline{b_1}=\overline{a_2}-\overline{b_2}$.\\
So it is well-defined.\\
\\
\textbf{Exercise \ref{exercise:EquivalenceRelationsChap:57}:}\\
a. Let define $f:Z_4\implies Z_{12}$ given by $f([a]_4)=[a]_{12}$.\\
If $[a]_{12}=[b]_{12} \implies a\equiv b\pmod{12} \implies a-b=3.4.k \implies a\equiv b\pmod{4}$\\
$\implies [a]_4=[b]_4 \implies f([a]_4)=f([b]_4)$.\\
\\
\textbf{Exercise \ref{exercise:EquivalenceRelationsChap:66}:}\\
a. $A=\{-3,-2,-1,0,1,2,3\}$\\
Reflexive: Let $x\in A \implies f(x)=x^2=f(x) \implies x\sim x$.\\
Symmetric: Let $x_1,x_2\in A$. If $x_1\sim x_2 \implies f(x_1)=x_1^2=f(x_2)=x_2^2$\\
$\implies f(x_2)=f(x_1) \implies x_2 \sim x_1$.\\
Transitive: transitive rule is not violated.\\
\\
\textbf{Exercise \ref{exercise:EquivalenceRelationsChap:67}:}\\
a. $x^2+y^2=r^2$\\
b. Suppose that $C_r\cap C_s \neq \emptyset$ where $s\neq r$ ($r,s>0$).\\
$\implies$  there is $(x_1,y_1)\in (C_r\cap C_s)$\\
$\implies x_1^2+y_1^2=r^2$ and $\implies x_1^2+y_1^2=s^2$\\
$\implies r^2=s^2 \implies r=s \to$ it's a contradiction to the supposition that $r\neq s$\\
$\implies C_r\cap C_s=\emptyset$.\\
c. Let $(x,y)\in R^2$\\
$\implies$  distance from point $(x,y)$ to the origin is $d=\sqrt{x^2+y^2}$\\
$\implies x^2+y^2=d^2 \implies (x,y)\in C_d$.\\
So each element of $R^2$ is in at least one circle.\\
d. From (c), we have $\bigcup_{r=0}^{\infty}C_r=R^2$\\
From (b), we have $C_r\cap C_s=\emptyset$ where $r\neq s$, which also means that $\bigcap_{r=0}^{\infty}C_r=\emptyset$\\
$\implies \{C_r|r\in [0,\infty)\}$ forms a partition of $R^2$ by definition of partition.\\
e. $(x_1,y_1)\sim (x_2,y_2)$ iff $x_1^2+y_1^2=x_2^2+y_2^2$\\


\section{Solutions for  ``Symmetries of Plane Figures''}
\noindent\textbf{\textit{ (Chapter \ref{symmetries})}}\bigskip

\textbf{Exercise \ref{exercise:symmetries:bijectnotsym}:}\\
It's not symmetry because it doesn't preserve the length and position of corresponding side $\overline{AB}$.\\
$\begin{pmatrix}
A & B & C & D & E & F\\
A & C & B & D & E & F
\end{pmatrix}$\\
\\
\textbf{Exercise \ref{exercise:symmetries:10}:}\\
b. square\\
c. $r_{360}$\\
d. $id: \begin{pmatrix}
A & B & C & D\\
A & C & B & D
\end{pmatrix}$
$\qquad r_{180}: \begin{pmatrix}
A & B & C & D\\
C & D & A & B
\end{pmatrix}$\\
\\
\textbf{Exercise \ref{exercise:symmetries:SymmComposition}:}\\
c. $r_{180}: \begin{pmatrix}
A & B & C & D\\
C & D & A & B
\end{pmatrix}$
$\qquad s_v: \begin{pmatrix}
A & B & C & D\\
D & C & B & A
\end{pmatrix}$\\
\\
$r_{180}\circ s_v=\begin{pmatrix}
(A,B) & (B,A) & (C,D) & (D,C)
\end{pmatrix}$\\
It's a symmetry $s_h$.\\
\\
d. $s_v\circ r_{180}=\begin{pmatrix}
(A,B) & (B,A) & (C,D) & (D,C)
\end{pmatrix}$\\
It's a symmetry $s_h$.\\
\\
\textbf{Exercise \ref{exercise:symmetries:CompSymm}:}\\
b. Yes, $r_{180}\circ s_v=s_h$\\
c. $s_h\circ s_v=\begin{pmatrix}
A & B & C & D\\
C & D & A & B
\end{pmatrix}$\\
It's a symmetry $r_{180}$\\
\\
\textbf{Exercise \ref{exercise:symmetries:16}:}\\
a. $f=r_{240}=\begin{pmatrix}
A & B & C & D & E & F\\
E & F & A & B & C & D
\end{pmatrix}$
$\qquad g=r_{120}=\begin{pmatrix}
A & B & C & D & E & F\\
C & D & E & F & A & B
\end{pmatrix}$\\
$f\circ g=\begin{pmatrix}
A & B & C & D & E & F\\
A & B & C & D & E & F
\end{pmatrix}=id$\\
$g\circ f=\begin{pmatrix}
A & B & C & D & E & F\\
A & B & C & D & E & F
\end{pmatrix}=id$\\
\\
b. $f=id=\begin{pmatrix}
A & B & C & D & E & F\\
A & B & C & D & E & F
\end{pmatrix}$
$\qquad g=r_{120}=\begin{pmatrix}
A & B & C & D & E & F\\
C & D & E & F & A & B
\end{pmatrix}$\\
$f\circ g=\begin{pmatrix}
A & B & C & D & E & F\\
C & D & E & F & A & B
\end{pmatrix}=r_{120}$\\
$g\circ f=\begin{pmatrix}
A & B & C & D & E & F\\
C & D & E & F & A & B
\end{pmatrix}=r_{120}$\\
\\
c. $f=r_{240}=\begin{pmatrix}
A & B & C & D & E & F\\
E & F & A & B & C & D
\end{pmatrix}$
$\qquad g=s_{BE}=\begin{pmatrix}
A & B & C & D & E & F\\
C & B & A & F & E & D
\end{pmatrix}$\\
$f\circ g=\begin{pmatrix}
A & B & C & D & E & F\\
A & F & E & D & C & B
\end{pmatrix}=s_{AD}$\\
$g\circ f=\begin{pmatrix}
A & B & C & D & E & F\\
E & D & C & B & A & F
\end{pmatrix}=s_{CF}$\\
\\
\textbf{Exercise \ref{exercise:symmetries:S3Table}:}\\
b. row 4, column 2\\
$\mu_1\circ \rho_1=\begin{pmatrix}
A & B & C\\
A & C & B
\end{pmatrix}\circ \begin{pmatrix}
A & B & C\\
B & C & A
\end{pmatrix}=\begin{pmatrix}
A & B & C\\
C & B & A
\end{pmatrix}=\mu_2$\\
\\
c. row 3, column 6\\
$\rho_2\circ \mu_3=\begin{pmatrix}
A & B & C\\
C & A & B
\end{pmatrix}\circ \begin{pmatrix}
A & B & C\\
B & A & C
\end{pmatrix}=\begin{pmatrix}
A & B & C\\
A & C & B
\end{pmatrix}=\mu_1$\\
\\
\textbf{Exercise \ref{exercise:symmetries:20}:}\\
a. Any symmetry composes with $id$ leave that symmetry unchanged.\\
$Id$ composes with any symmetry still leave that symmetry unchanged.\\
b. Yes.\\
Inverse of $id$ is $id$.\\
Inverse of $\rho_1$ is $\rho_2$.\\
Inverse of $\rho_2$ is $\rho_1$.\\
Inverse of $\mu_1$ is $\mu_1$.\\
Inverse of $\mu_2$ is $\mu_2$.\\
Inverse of $\mu_3$ is $\mu_3$.\\
c. because $\mu_1\circ \mu_2 = \rho_1$ but $\mu_2\circ \mu_1=\rho_2$\\
\\
\textbf{Exercise \ref{exercise:symmetries:InverseRot}:}\\
a. $r^k=r\circ r\circ\dots \circ r$ (k times)\\
$r^m=r\circ r\circ r\circ\dots \circ r$ (m times)\\
$\implies r^k\circ r^m=(r\circ r\circ\dots \circ r)\circ(r\circ r\circ r\circ\dots \circ r)=r^{k+m}$\\
\\
b. $r^k\circ r^{n-k}=r^{k+n-k}=r^n=id$\\
$r^{n-k}\circ r^{k}=r^{n-k+k}=r^n=id$\\
\\
c. $(r^k)^{-1}=r^{n-k}$\\
\\
\textbf{Exercise \ref{exercise:symmetries:31}:}\\
b. Two vertices are fixed: 1 and 5.\\
c. $s^2=s\circ s=id$\\
\\
\textbf{Exercise \ref{exercise:symmetries:33}:}\\
Suppose $0<q<n$ and $s=s\circ r^q$, we have:
$s\circ s=s\circ(s\circ r^q)$ (compose s to the left of both sides)\\
$\implies s\circ s=(s\circ s)\circ r^q$ (associative property)\\
$\implies id=id\circ r^q$ (because $s\circ s=id$)\\
$\implies id=r^q$ ($id$ is group identity)\\
It's a contradiction since we have shown that $id\neq r^q$ when $0<q<n$.\\
Therefore, $s\neq s\circ r^q$.\\
\\
\textbf{Exercise \ref{exercise:symmetries:34}:}\\
...we suppose $s\circ r^p=r^q$.\\
...we obtain $(s\circ r^p)\circ r^{n-p}=r^q\circ r^{n-p}$.\\
By associativity, we have $s\circ (r^p\circ r^{n-p})=r^q\circ r^{n-p}$.\\
...that $r^p\circ r^{n-p}=id$, we obtain $s=r^q\circ r^{n-p}$.\\
...the right side is a rotation.\\
...we conclude $s\circ r^p\neq r^q$.\\
\\
\textbf{Exercise \ref{exercise:symmetries:36}:}\\
a. Reflection about diagonal 13.\\
Reflection about diagonal 24.\\
Reflection about horizontal axis of symmetry.\\
Reflection about vertical axis of symmetry.\\
b. $\mu\circ \mu = id$.\\
\\
\textbf{Exercise \ref{exercise:symmetries:HexagonRefl}:}\\
a. $s_1=\begin{pmatrix}
1 & 2 & 3 & 4 & 5 & 6\\
1 & 6 & 5 & 4 & 3 & 2
\end{pmatrix}$
\\
$s_2=\begin{pmatrix}
1 & 2 & 3 & 4 & 5 & 6\\
3 & 2 & 1 & 6 & 5 & 4
\end{pmatrix}$
$\qquad s_3=\begin{pmatrix}
1 & 2 & 3 & 4 & 5 & 6\\
5 & 4 & 3 & 2 & 1 & 6
\end{pmatrix}$\\
\\
$s_{12}=\begin{pmatrix}
1 & 2 & 3 & 4 & 5 & 6\\
2 & 1 & 6 & 5 & 4 & 3
\end{pmatrix}$
$\qquad s_{23}=\begin{pmatrix}
1 & 2 & 3 & 4 & 5 & 6\\
4 & 3 & 2 & 1 & 6 & 5
\end{pmatrix}$\\
\\
$s_{34}=\begin{pmatrix}
1 & 2 & 3 & 4 & 5 & 6\\
6 & 5 & 4 & 3 & 2 & 1
\end{pmatrix}$\\
\\
b. $\mu\circ\mu=id$\\
c. 3 reflections have no fixed vertices.\\
\\
\textbf{Exercise \ref{exercise:symmetries:nonagon}:}\\
c. $\mu_3=\begin{pmatrix}
1 & 2 & 3 & 4 & 5 & 6 & 7 & 8 & 9 & 10\\
2 & 1 & 10 & 9 & 8 & 7 & 6 & 5 & 4 & 3
\end{pmatrix}$\\
\\
d. $\mu_1\circ\mu_1=id$\\
$\mu_2\circ\mu_2=id$\\
$\mu_3\circ\mu_3=id$\\
\\
\textbf{Exercise \ref{exercise:symmetries:41}:}\\
a. $s\circ r^p$ is either a rotation or a reflection. (Proposition 42)\\
Suppose $s\circ r^p$ is a rotation.\\
Then $s\circ r^p=r^k$ $(k\in Z)$.\\
Multiply both sides on the right by $r^{n-p}$\\
Then $s\circ r^p\circ r^{n-p}=r^k\circ r^{n-p}$\\
$\implies s=r^{k+n-p}$\\
This would mean $s$ is a rotation, which is a contradiction.\\
Therefore, $s\circ r^p$ must be a reflection.\\
\\
\textbf{Exercise \ref{exercise:symmetries:44}:}\\
\begin{center}
	\begin{tabular}{c| c c c c c c c c }
		o & $id$ & $r$ & $r^2$ & $r^3$ & $s$ & $s\circ r$ & $s\circ r^2$ & $s\circ r^3$\\
		\hline
		$id$ & $id$ & $r$ & $r^2$ & $r^3$ & $s$ & $s\circ r$ & $s\circ r^2$ & $s\circ r^3$\\
		$r$ & $r$ & $r^2$ & $r^3$ & $id$ & $s\circ r^3$ & $s$ & $s\circ r$ & $s\circ r^2$\\
		$r^2$ & $r^2$ & $r^3$ & $id$ & $r^2$ & $s\circ r^2$ & $s\circ r^3$ & $s$ & $s\circ r$\\
		$r^3$ & $r^3$ & $id$ & $r$ & $r^2$ & $s\circ r$ & $s\circ r^2$ & $s\circ r^3$ & $s$\\
		$s$ & $s$ & $s\circ r$ & $s\circ r^2$ & $s\circ r^3$ & $id$ & $r$ & $r^2$ & $r^3$\\
		$s\circ r$ & $s\circ r$ & $s\circ r^2$ & $s\circ r^3$ & $s$ & $r^3$ & $id$ & $r$ & $r^2$\\
		$s\circ r^2$ & $s\circ r^2$ & $s\circ r^3$ & $s$ & $s\circ r$ & $r^2$ & $r^3$ & $id$ & $r$\\
		$s\circ r^3$ & $s\circ r^3$ & $s$ & $s\circ r$ & $s\circ r^2$ & $r$ & $r^2$ & $r^3$ & $id$
	\end{tabular}
\end{center}
\hfill\\
\hfill\\
\textbf{Exercise \ref{exercise:symmetries:46}:}\\
b.
\begin{center}
	\begin{tabular}{c| c c c c}
		o & $id$ & $r$ & $r^2$ & $r^3$\\
		\hline
		$id$ & $id$ & $r$ & $r^2$ & $r^3$\\
		$r$ & $r$ & $r^2$ & $r^3$ & $id$\\
		$r^2$ & $r^2$ & $r^3$ & $id$ & $r$\\
		$r^3$ & $r^3$ & $id$ & $r$ & $r^2$
	\end{tabular}
\end{center}
c.
\begin{center}
	\begin{tabular}{c| c c c c}
		o & $1$ & $z$ & $z^2$ & $z^3$\\
		\hline
		$1$ & $1$ & $z$ & $z^2$ & $z^3$\\
		$z$ & $z$ & $z^2$ & $z^3$ & $1$\\
		$z^2$ & $z^2$ & $z^3$ & $1$ & $z$\\
		$z^3$ & $z^3$ & $1$ & $z$ & $z^2$
	\end{tabular}
\end{center}

\section{Solutions for ``Permutations''}
\noindent\textbf{\textit{ (Chapter \ref{permute})}}\bigskip

\noindent\textbf{Exercise \ref{exercise:permute:S_4}:}\\
c. $X=\{A,B,C,D\}$\\
Permutations that are not symmetries of the square:\\
$\begin{pmatrix}
A & B & C & D\\
A & C & B & D
\end{pmatrix}$
$\qquad\begin{pmatrix}
A & B & C & D\\
A & C & D & B
\end{pmatrix}$
$\qquad\begin{pmatrix}
A & B & C & D\\
A & B & D & C
\end{pmatrix}$
$\qquad\begin{pmatrix}
A & B & C & D\\
A & D & B & C
\end{pmatrix}$\\
\\
$\begin{pmatrix}
A & B & C & D\\
B & A & C & D
\end{pmatrix}$
$\qquad\begin{pmatrix}
A & B & C & D\\
B & C & A & D
\end{pmatrix}$
$\qquad\begin{pmatrix}
A & B & C & D\\
B & D & A & C
\end{pmatrix}$
$\qquad\begin{pmatrix}
A & B & C & D\\
B & D & C & A
\end{pmatrix}$\\
\\
$\begin{pmatrix}
A & B & C & D\\
C & A & D & B
\end{pmatrix}$
$\qquad\begin{pmatrix}
A & B & C & D\\
C & A & B & D
\end{pmatrix}$
$\qquad\begin{pmatrix}
A & B & C & D\\
C & B & D & A
\end{pmatrix}$
$\qquad\begin{pmatrix}
A & B & C & D\\
C & D & B & A
\end{pmatrix}$\\
\\
$\begin{pmatrix}
A & B & C & D\\
D & B & A & C
\end{pmatrix}$
$\qquad\begin{pmatrix}
A & B & C & D\\
D & B & C & A
\end{pmatrix}$
$\qquad\begin{pmatrix}
A & B & C & D\\
D & C & A & B
\end{pmatrix}$
$\qquad\begin{pmatrix}
A & B & C & D\\
D & A & B & C
\end{pmatrix}$\\
\\
d. Additional elements in $S_x$ are not symmetries of the rectangle:\\
$\begin{pmatrix}
A & B & C & D\\
C & B & A & D
\end{pmatrix}$
$\qquad\begin{pmatrix}
A & B & C & D\\
A & D & C & B
\end{pmatrix}$
$\qquad\begin{pmatrix}
A & B & C & D\\
B & C & D & A
\end{pmatrix}$
$\qquad\begin{pmatrix}
A & B & C & D\\
D & A & B & C
\end{pmatrix}$\\
\\

\textbf{Exercise \ref{exercise:permute:7}:}\\
a. $\mu\circ\sigma=\begin{pmatrix}
A & B & C & D\\
D & C & B & A
\end{pmatrix}\circ\begin{pmatrix}
A & B & C & D\\
C & D & A & B
\end{pmatrix}=\begin{pmatrix}
A & B & C & D\\
B & A & D & C
\end{pmatrix}$\\
\\
b. $\tau\circ\rho=\begin{pmatrix}
1 & 2 & 3 & 4\\
4 & 3 & 2 & 1
\end{pmatrix}\circ\begin{pmatrix}
1 & 2 & 3 & 4\\
3 & 4 & 1 & 2
\end{pmatrix}=\begin{pmatrix}
1 & 2 & 3 & 4\\
2 & 1 & 4 & 3
\end{pmatrix}$\\
\\
c. Yes, $\mu\circ\sigma$ is equivalent to $\tau\circ\rho$ because if we rename $A,B,C,D$ by $1,2,3,4$, we will get the same result for both compositions.\\
\\
\textbf{Exercise \ref{exercise:permute:11}:}\\
a. $|S_X|=6$; $|S_Y|=6$\\
$\sigma_1=\begin{pmatrix}
A & B & C\\
A & B & C
\end{pmatrix}$
$\qquad\tau_1=\begin{pmatrix}
M & N & P\\
M & N & P
\end{pmatrix}$\\
\\
$\sigma_2=\begin{pmatrix}
A & B & C\\
B & C & A
\end{pmatrix}$
$\qquad\tau_2=\begin{pmatrix}
M & N & P\\
N & P & M
\end{pmatrix}$\\
\\
$\sigma_3=\begin{pmatrix}
A & B & C\\
C & A & B
\end{pmatrix}$
$\qquad\tau_3=\begin{pmatrix}
M & N & P\\
P & M & N
\end{pmatrix}$\\
\\
$\sigma_4=\begin{pmatrix}
A & B & C\\
A & C & B
\end{pmatrix}$
$\qquad\tau_4=\begin{pmatrix}
M & N & P\\
M & P & N
\end{pmatrix}$\\
\\
$\sigma_5=\begin{pmatrix}
A & B & C\\
B & A & C
\end{pmatrix}$
$\qquad\tau_5=\begin{pmatrix}
M & N & P\\
N & M & P
\end{pmatrix}$\\
\\
$\sigma_6=\begin{pmatrix}
A & B & C\\
C & B & A
\end{pmatrix}$
$\qquad\tau_6=\begin{pmatrix}
M & N & P\\
P & N & M
\end{pmatrix}$\\
\\
\begin{center}
	\begin{tabular}{c| c c c c c c}
		o & $\sigma_1$ & $\sigma_2$ & $\sigma_3$ & $\sigma_4$ & $\sigma_5$ & $\sigma_6$\\
		\hline
		$\sigma_1$ & $\sigma_1$ & $\sigma_2$ & $\sigma_3$ & $\sigma_4$ & $\sigma_5$ & $\sigma_6$\\
		$\sigma_2$ & $\sigma_2$ & $\sigma_3$ & $\sigma_1$ & $\sigma_5$ & $\sigma_6$ & $\sigma_4$\\
		$\sigma_3$ & $\sigma_3$ & $\sigma_1$ & $\sigma_2$ & $\sigma_6$ & $\sigma_4$ & $\sigma_5$\\
		$\sigma_4$ & $\sigma_4$ & $\sigma_6$ & $\sigma_5$ & $\sigma_1$ & $\sigma_3$ & $\sigma_2$\\
		$\sigma_5$ & $\sigma_5$ & $\sigma_4$ & $\sigma_6$ & $\sigma_2$ & $\sigma_1$ & $\sigma_3$\\
		$\sigma_6$ & $\sigma_6$ & $\sigma_5$ & $\sigma_4$ & $\sigma_3$ & $\sigma_2$ & $\sigma_1$\\
	\end{tabular}
\end{center}
\hfill\\
b. Bijection from $X\implies Y$:\\
$f=\begin{pmatrix}
A & B & C\\
M & N & P
\end{pmatrix}$\\
\\
Bijection from $S_X$ to $S_Y=\begin{pmatrix}
\sigma_1 & \sigma_2 & \sigma_3 & \sigma_4 & \sigma_5 & \sigma_6\\
\tau_1 & \tau_2 & \tau_3 & \tau_4 & \tau_5 & \tau_6
\end{pmatrix}$\\
\\
\textbf{Exercise \ref{exercise:permute:permute_S4}:}\\
a.
\begin{center}
	\begin{tabular}{c| c c c c}
		o & $id$ & $\sigma$ & $\tau$ & $\mu$\\
		\hline
		$id$ & $id$ & $\sigma$ & $\tau$ & $\mu$\\
		$\sigma$ & $\sigma$ & $\mu$ & $\sigma\circ\tau$ & $id$\\
		$\tau$ & $\tau$ & $\tau\circ\sigma$ & $id$ & $\tau\circ\mu$\\
		$\mu$ & $\mu$ & $id$ & $\mu\circ\tau$ & $\sigma$\\
	\end{tabular}
\end{center}
b. G is a subgroup since it has identity element and each item has its inverse.\\
\\
\textbf{Exercise \ref{exercise:permute:long_cycle_perm1}:}\\
$(135624)$\\
\\
\textbf{Exercise \ref{exercise:permute:long_cycle_perm3}:}\\
a. $\mu=\begin{pmatrix}
1 & 2 & 3 & 4 & 5 & 6\\
6 & 1 & 2 & 3 & 4 & 5
\end{pmatrix}$\\
b. $(165432)$\\
c. $\rho$ expresses a reverse order of $\mu$\\
\\
\textbf{Exercise \ref{exercise:permute:perm_not_comm}:}\\
$\sigma=(1532)$ and $\tau=(126)$\\
a. $\tau\sigma=(1536)$\\
b. $\sigma\tau=(2653)$\\
Permutations are not commutative.\\
\\
\textbf{Exercise \ref{exercise:permute:cycle_comp_exer1}:}\\
b. $\sigma\delta=(347)(135)=(14735)$\\
c. $\delta\rho=(135)(567)=(13567)$\\
d. $\rho\delta=(567)(135)=(13675)$\\
\\
\textbf{Exercise \ref{exercise:permute:tab_to_cycle}:}\\
a. $\rho=(12)(35)(46)$\\
b. $\sigma=(125)(36)$\\
c. $\omega=(1463)(25)$\\
\\
\textbf{Exercise \ref{exercise:permute:cycle_to_tab}:}\\
a. $\mu=\begin{pmatrix}
1 & 2 & 3 & 4 & 5 & 6 & 7 & 8 & 9\\
1 & 5 & 4 & 7 & 9 & 6 & 3 & 8 & 2
\end{pmatrix}$\\
\\
b. $\sigma=\begin{pmatrix}
1 & 2 & 3 & 4 & 5 & 6 & 7 & 8 & 9\\
4 & 5 & 9 & 1 & 6 & 7 & 8 & 2 & 3
\end{pmatrix}$\\
\\
\textbf{Exercise \ref{exercise:permute:D4_cycles}:}\\
Symmetries of a square:\\
$(12)(34)\qquad\qquad(1432)\qquad\qquad(14)(23)\qquad\qquad(13)(24)$\\
$(24)\qquad\qquad(1234)\qquad\qquad(13)\qquad\qquad id$\\
\\
\textbf{Exercise \ref{exercise:permute:dis_cycles_comm}:}\\
a. $\begin{pmatrix}
1 & 2 & 3 & 4 & 5 & 6\\
2 & 3 & 1 & 5 & 4 & 6
\end{pmatrix}=(123)(45)$\\
\\
$(45)(123)=\begin{pmatrix}
1 & 2 & 3 & 4 & 5 & 6\\
2 & 3 & 1 & 5 & 4 & 6
\end{pmatrix}$\\
\\
b. $(14)(263)=\begin{pmatrix}
1 & 2 & 3 & 4 & 5 & 6\\
4 & 6 & 2 & 1 & 5 & 3
\end{pmatrix}$\\
\\
$(263)(14)=\begin{pmatrix}
1 & 2 & 3 & 4 & 5 & 6\\
4 & 6 & 2 & 1 & 5 & 3
\end{pmatrix}$\\
\\
e. Product of disjoint cycles is commutative.\\
\\
\textbf{Exercise \ref{exercise:permute:disjoint_commutative}:}\\
$\sigma\tau(x)=\tau\sigma(x)$\\
$A\cap B=\emptyset$\\
$x\in A$ and $x\not\in B$\\
$x\not\in A$ and $x\not\in B$\\
(ii) $x\not\in A$ ... that $\sigma(x)=x$\\
$\tau\sigma(x)=\tau(\sigma(x))=\tau(x)$\\
$x\in B$ ... $\tau(x)\in B$ ... $\tau(x)\not\in A$\\
$\sigma\tau(x)=\sigma(\tau(x))=\tau(x)$\\
It follows that $\sigma\tau(x)=\tau\sigma(x)$\\
(iii) $\sigma(x)=x$\\
$x\not\in B \implies \tau(x)=x$\\
$\tau\sigma(x)=x$ and $\sigma\tau(x)=x$\\
$\sigma\tau(x)=\tau\sigma(x)$\\
\\
\textbf{Exercise \ref{exercise:permute:cycle_comp_5}:}\\
a. $\sigma\tau=[(1257)(34)][(265)(137)]=(143)(267)$\\
d. $\sigma\rho=(1257)(34)(135)(246)(78)=(14652378)$\\
\\
\textbf{Exercise \ref{exercise:permute:cycle_comp_2}:}\\
a. $(1345)(234)=(135)(24)$\\
b. $(12)(1253)=(253)$\\
g. $(1254)^2(123)(45)=(14)(235)$\\
\\
\textbf{Exercise \ref{exercise:permute:cycle_types}:}\\
a. $S_6$\\
Types of permutations in $S_6$ are:\\
- The identity\\
- Single cycles of lengths 6, 5, 4, 3, and 2.\\
- Two disjoint cycles of lengths 2 and 4; two disjoint cycles of lengths 2 and 3; two disjoint cycles of lengths 2 and 2; two disjoint cycles of lengths 3 and 3.\\
- Three disjoint cycles of lengths 2 and 2 and 2.\\
b. Type of permutations in $S_7$ are:\\
- The identity\\
- Single cycles of lengths 7, 6, 5, 4, 3, and 2.\\
- Two disjoint cycles of lengths: 2 and 5; 2 and 4; 2 and 3; 2 and 2; 3 and 4; 3 and 3.\\
- Three disjoint cycles of lengths: 2 and 2 and 2; 2 and 2 and 3.\\
\\
\textbf{Exercise \ref{exercise:permute:6th_power}:}\\
a. $(125843)^2=(154)(283)$\\
c. $(125843)^4=(145)(238)$\\
\\
\textbf{Exercise \ref{exercise:permute:power of n-cycle1}:}\\
a. $\sigma(2)=5\qquad\qquad\sigma^2(2)=7\qquad\qquad\sigma^3(2)=8\qquad\qquad\sigma^4(2)=2$\\
$\sigma^{3482991}(2)=\sigma^3(2)=8$\\
b. $\sigma(5)=7\qquad\qquad\sigma^2(5)=8\qquad\qquad\sigma^3(5)=2\qquad\qquad\sigma^4(5)=5$\\
$\sigma^{3482991}(5)=\sigma^3(5)=2$\\
c. $\sigma^k(x)=\sigma^{\mod(k,4)}(x)$ for $x\in A$\\
e. $\sigma^k(x)=x$ for $x\in X\setminus A$\\
f. $2\not\in K$, $3\not\in K$, $4\in K$\\
If $\mod(k,4)=0$ then $k\in K$.\\
\\
\textbf{Exercise \ref{exercise:permute:length_equals_order}:}\\
$\sigma^k(x)=x\qquad\forall x\in X$\\
(i) $\sigma(x)=x$\\
$\sigma^2(x)=\sigma(\sigma(x))=\sigma(x)=x$\\
$\sigma^3(x)=x$\\
$\sigma^k(x)=x$ for any natural number $k$\\
(ii) $1\le j\le k$\\
$\sigma(x)=\sigma(a_j)=a_{j+1\pmod{k}}$\\
$\sigma^2(x)=\sigma(a_{j+1})=a_{j+2\pmod{k}}$\\
$\sigma^k(x)=a_{j+k\pmod{k}}=a_j=x$\\
$\forall x\in X$, $\sigma^k(x)=x$.\\
It follows that $\sigma^k=id$.\\
\\
\textbf{Exercise \ref{exercise:permute:cycle_order_1}:}\\
a. $(1264)^{11}=[(1264)^4]^2(1264)^3=id(1264)^3=(1462)$\\
\\
b. $(125843)^{53}=(125843)^{mod(53,6)}=(125843)^5=(134852)$\\
\\
c. $(352)(136)(1254)^{102}=(352)(136)(1254)^{mod(102,4)}=(12436)$\\
\\
\textbf{Exercise \ref{exercise:permute:64}:}\\
$b\in Z_k$\\
$\tau^l=\tau^{ak+b}=r^{ak}r^b=(r^k)^ar^b=id^ar^b=r^b$\\
$r^b=id$\\
$b<k$\\
k is the smallest integer such that $r^k=id$\\
$b\equiv 0\pmod{k}$\\
$l\equiv 0\pmod{k}$\\
\\
\textbf{Exercise \ref{exercise:permute:k_power_disjoint_cycles}:}\\
a. $(\sigma\tau)^2=(\sigma\tau)(\sigma\tau)$\\
$=\sigma(\tau\sigma)\tau$ (associative)\\
$=\sigma(\sigma\tau)\tau$ (commutative)\\
$=(\sigma\sigma)(\tau\tau)$ (associative)\\
$=\sigma^2\tau^2$\\
\\
\textbf{Exercise \ref{exercise:permute:Sn_orders}:}\\
a. Possible orders of $S_6$ are: 2, 3, 4, 5, 6.\\
b. Possible orders of $S_7$ are: 2, 3, 4, 5, 6, 7, 10, 12.\\
\\
\textbf{Exercise \ref{exercise:permute:example_transpositions}:}\\
b. $(14)(18)(19)=(1984)$\\
c. $(16)(15)(14)(13)(12)=(123456)$\\
\\
\textbf{Exercise \ref{exercise:permute:example_transpositions2}:}\\
a. $(1492)=(12)(19)(14)$\\
b. $(12345)=(15)(14)(13)(12)$\\
c. $(472563)=(43)(46)(45)(42)(47)$\\
\\
\textbf{Exercise \ref{exercise:permute:prod_of_trans_1}:}\\
a. $(14356)=(16)(15)(13)(14)$\\
b. $(156)(234)=(16)(15)(24)(23)$\\
\\
\textbf{Exercise \ref{exercise:permute:inv_comps}:}\\
a. $(12537)^{-1}=(17352)$\\
b. $[(12)(34)(12)(47)]^{-1}=(47)(12)(34)(12)=(374)$\\
c. $[(1235)(467)]^{-2}=[(467)^{-1}(1235)^{-1}]^2=[(476)(1532)]^2=(467)(13)(25)$\\
\\
\textbf{Exercise  \ref{ex:evenoddprod}:}\\
%%% Note for some reason the conventional label doesn't work
An even permutation is any permutation can be written as a product of an even number of transpositions.\\
\begin{enumerate}[(a)]
\item
Suppose $\sigma$ and $\tau$ are even permutations, then:
\begin{itemize}
\item
$\sigma$ is a product of $2m$ transpositions.
\item
$\tau$ is a product of $2n$ transpositions.
\item
So $\sigma\tau$ is a product of $(2m+2n)$ transpositions.
\item
$\implies$  it's an even permutation.
\end{itemize}
\item
 Suppose $\sigma$ and $\mu$ are odd permutations, then:
\begin{itemize}
\item
$\sigma$ is a product of $2k+1$ transpositions.
\item
$\mu$ is a product of $2l+1$ transpositions.
\item
So $\sigma\mu$ is a product of $2(l+k+1)$ transpositions.
\item
$\implies$  it's an even permutation.
\end{itemize}
\item
 Let $\tau$ is an even permutation $\implies$  it's a product of $2k$ transpositions.\\
Let $\mu$ is an odd permutation $\implies$  it's a product of $2l+1$ transpositions.\\
$\implies \tau\mu$ is a product of $2(k+l)+1$ transpositions.\\
$\implies$  it's an odd permutation $\implies$  its parity is 1.\\
Since product of permutations is commutative, product of an odd permutation and an even permutation also has parity 1.\\
\end{enumerate}

\section{Solutions for  ``Abstract Groups''}
\noindent\textbf{\textit{ (Chapter \ref{groups})}}\bigskip

\noindent\textbf{Exercise \ref{exercise:groups:Rstar_group}:}\\
a. $(R^*,.)$ is a group.\\
$R^*$ is closed under multiplication.\\
$R^*$ has 1 as the identity element.\\
Each element has its own multiplicative inverse.\\
Multiplication is associative.\\
b. $(R^*,+)$ is not a group because it's not closed under addition. For example, we have $1+(-1)=0\not\in R^*$.\\
c. $(R^*,.)$ has infinite order.\\
\\
\textbf{Exercise \ref{exercise:groups:C_star}:}\\
a. $C^*$ is not a group under addition because it's not closed. For example $3i+(-3i)=0\not\in C^*$.\\
b. $C^*$ is a group under multiplication.\\
$C^*$ is closed under multiplication.\\
$1\in C$ is the identity element.\\
Each element in $C^*$ has its own inverse.\\
Multiplication is associative.\\
c. $|(C^*,.)|$ has infinite order.\\
\\
\textbf{Exercise \ref{exercise:groups:15}:}\\
a. Because 0 doesn't have its inverse.\\
Exception: $S=\{0\}$ is a group under multiplication.\\
b. Because $S$ doesn't have the identity element.\\
\\
\textbf{Exercise \ref{exercise:groups:S_group}:}\\
- Identity element:\\
Suppose $a,e\in S$: $a*e=a+e+ae=a$\\
$\implies e+ae=0$\\
$\implies e(1+a)=0$\\
$\implies e=0$ (because $S=R\setminus\{-1\}$)\\
$\implies e=0$ is the identity element.\\
- Inverse: Suppose $a\in S$: $a*a^{-1}=0$\\
$\implies a+a^{-1}+aa^{-1}=0$\\
$\implies a^{-1}=\displaystyle\frac{-a}{1+a}\neq -1$\\
$\implies$  each element has its own inverse.\\
- Associative:\\
$(a*b)*c=(a+b+ab)*c=(a+b+ab+c)+(a+b+ab)c$\\
$a*(b*c)=a*(b+c+bc)=(a+b+c+bc)+a(b+c+bc)=(a+b+ab+c)+(a+b+ab)c$\\
$\implies (a*b)*c=a*(b*c)$\\
$\implies (S,*)$ is an abelian group.\\
\\
\textbf{Exercise \ref{exercise:groups:integer_coordinate_plane}:}\\
a.\\
- identity element: $(0,0)$\\
$(a,b)\circ (0,0)=(a+0,b+0)=(a,b)$\\
- inverse: $(a,b)\circ (-a,-b)=(0,0)$\\
- closed: $(a,b)\circ (c,d)=(a+c,b+d)\in Z$x$Z$\\
- commutative: $(a,b)\circ (c,d)=(a+c,b+d)=(c+a,d+b)=(c,d)\circ (a,b)$\\
- associative:\\
$[(a,b)\circ (c,d)]\circ (e,f)=(a+c,b+d)\circ (e,f)=(a+c+e,b+d+f)$\\
$(a,b)\circ[(c,d)\circ (e,f)]=(a,b)\circ (c+e,d+f)=(a+c+e,b+d+f)$\\
$\implies (H,\circ)$ is a group.\\
$\implies (H,\circ)$ is an abelian group because it has commutative property.\\
b. Similar to part (a), we have $(H,\circ$ is an abelian group because it has identity element, inverse, and is closed, associative, and commutative.\\
\\
\textbf{Exercise \ref{exercise:groups:19}:}
\begin{enumerate}[(a)]
\item
$(a,m)\circ (b,n)=(a+b,m+n)$\\
$(G,\circ)$ is not a group since it doesn't have identity element.
\item
$(G,\circ)$ is not a group because it doesn't satisfy the inverse element requirement.
\item
$(G,\circ)$ is an abelian group. Similar to Exercise \ref{exercise:groups:integer_coordinate_plane}, we see that $(G,\circ)$ satisfies all 4 requirements to be a group and it's also commutative.
\end{enumerate}
%\textbf{Exercise 23:}\\
%$e$ is identity element\\
%If $g=e$, we have $g\circ g=g\circ e=g$.\\
%If $g\circ g=g$, we have $g\circ g\circ g^{-1}=g\circ g^{-1}\implies g\circ e=e\implies g=e$\\
%Therefore $g=e$ if and only if $g\circ g=g$\\
\textbf{Exercise \ref{exercise:groups:Cayley_groups2}:}
\begin{enumerate}[(a)]
\item

\begin{center}
	\begin{tabular}{c| c c c c }
		$\circ$ & a & b & c & d\\
		\hline
		a & a & b & c & d\\
		b & b & a & d & c\\
		c & c & d & a & b\\
		d & d & c & b & a
	\end{tabular}
\end{center}
\item

\begin{center}
	\begin{tabular}{c| c c c c }
		$\circ$ & a & b & c & d\\
		\hline
		a & c & a & d & b\\
		b & a & b & c & d\\
		c & d & c & b & a\\
		d & b & d & a & c
	\end{tabular}
\end{center}
\item

\begin{center}
	\begin{tabular}{c| c c c c }
		$\circ$ & a & b & c & d\\
		\hline
		a & d & c & b & a\\
		b & c & d & a & b\\
		c & b & a & d & c\\
		d & a & b & c & d
	\end{tabular}
\end{center}
\item
 
\begin{center}
	\begin{tabular}{c| c c c c }
		$\circ$ & a & b & c & d\\
		\hline
		a & a & b & c & d\\
		b & b & a & d & c\\
		c & c & d & b & a\\
		d & d & c & a & b
	\end{tabular}
\end{center}
\end{enumerate}
\textbf{Exercise \ref{exercise:groups:Cayley_groups3}:}
\begin{enumerate}
\item
We have $b\circ a=b\implies a=e$. However, $c\circ a=d\implies a\neq e$.
So we can't fill the table to make a group.
\item
 We have $a\circ a=a\implies a=e$ and $b\circ b=b\implies b=e$\\
$\implies$  can't fill the table to make a group because there are 2 different $e$.
\item
There is no identity element.
\end{enumerate}
\textbf{Exercise \ref{exercise:groups:27}:}\\
- identity is 1\\
- each element has its own inverse\\
- closed\\
- commutative\\
- associative\\
$\implies$  it's a group.\\
\\
\textbf{Exercise \ref{exercise:groups:28}:}\\
It's abelian because $a\odot b=b\odot a$ for all elements in the group.\\
\\
\textbf{Exercise \ref{exercise:groups:U(n)_abgroup}:}\\
b. $m\in\bigcup(n)$ and $x$ is the inverse of $m$\\
$xm\equiv 1 \pmod{n}$\\
c. $xm=kn+1\implies x=\displaystyle\frac{kn+1}{m}$\\
$m$ is relative prime to $n\to$ gcd of $m$ and $n$ is 1.\\
1 is a multiple of 1 $\implies$  there is a solution.\\
\\
\textbf{Exercise \ref{exercise:groups:UnAbel}:}\\
$\bigcup(n)$ is abelian\\
Let $x,y\in\bigcup(n)$, we have $x\odot y\equiv a\pmod{n}$\\
$\implies xy=kn+a\implies yx=kn+a\implies y\odot x\equiv a\pmod{n}$\\
$\implies x\odot y=y\odot x$\\
\\
\textbf{Exercise \ref{exercise:groups:39}:}\\
a. $a^{-1}=\displaystyle\frac{5-3i}{34}$\\
\\
c. idendity is $(1,0)$\\
$(3,2)\circ(x,y)=(1,0)\implies y=-2$ and $x=\displaystyle\frac{1}{3}$\\
\\
\textbf{Exercise \ref{exercise:groups:42}:}\\
$(b^{-1}a^{-1}(ab)=b^{-1}(a^{-1}a)b=b^{-1}eb=b^{-1}b=e$\\
\\
\textbf{Exercise \ref{exercise:groups:44}:}\\
$a(a^{-1}(a^{-1})^{-1})=ae$ (def. of inverse)\\
$(aa^{-1})(a^{-1})^{-1}=ae$ (associative)\\
$e(a^{-1})^{-1}=ae$ (def. of inverse)\\
$(a^{-1})^{-1}=a$ (def. of identity)\\
\\
\textbf{Exercise \ref{exercise:groups:48}:}\\
a. $a^{-1}(ax)=a^{-1}b$\\
$(a^{-1}a)x=a^{-1}b$ (associative)\\
$ex=a^{-1}b$ (def. of inverse)\\
$x=a^{-1}b$ (def. of identity)\\
\\
\textbf{Exercise \ref{exercise:groups:49}:}\\
a. $ax=b$\\
$a^{-1}ax=a^{-1}b$\\
$x=a^{-1}b=-\displaystyle\frac{5}{3}+\displaystyle\frac{7}{3}i$
\\
\textbf{Exercise \ref{exercise:groups:51}:}\\
a. $\rho x=\mu$\\
$\implies x=\rho^{-1}\mu=(18752)(346)$\\
b. $x=\mu\rho^{-1}=(142)(5687)$\\
\\
\textbf{Exercise \ref{exercise:groups:53}:}\\
a. $X=A^{-1}B=\begin{pmatrix}
-10.75 & -6\\
12.5 & 7
\end{pmatrix}$\\
b. $X=BA^{-1}=\begin{pmatrix}
-1 & 1\\
3 & -2.75
\end{pmatrix}$\\
\\
\textbf{Exercise \ref{exercise:groups:56}:}\\
a. Because $ab\neq ba$ in most of groups.\\
$ba=ab$ only in abelian group.\\
b. $ba=ca$\\
$baa^{-1}=caa^{-1}$\\
$be=ce$\\
$b=c$\\
If $ab=ac$:\\
$\implies a^{-1}ab=a^{-1}ac$\\
$\implies b=c$\\
\\
\textbf{Exercise \ref{exercise:groups:59}:}\\
3. $(h^{-1}g^{-1})^{-n}=(h^{-1}g^{-1})^{-1}(h^{-1}g^{-1})^{-1}...(h^{-1}g^{-1})^{-1}$ (n times)\\
$=[(g^{-1})^{-1}(h^{-1})^{-1}][(g^{-1})^{-1}(h^{-1})^{-1}]...[(g^{-1})^{-1}(h^{-1})^{-1}]$ (n times)\\
$=(gh)(gh)...(gh)$ (n times)\\
$=(gh)^n$\\
\\
\textbf{Exercise \ref{exercise:groups:64}:}\\
a. $GL_2(R)$ is group of all invertible 2x2 matrices with multiplication operation.\\
$M_2(R)$ is group of all 2x2 matrices with addition operation.\\
$\implies GL_2(R)$ is not a subgroup of $M_2(R)$ because they have different group operations.\\
b. $\bigcup(n)$ is not a subgroup of $Z_n$ because they have different group operations.\\
$\bigcup(n)$ is a group under multiplication.\\
$Z_n$ is a group under addition.\\
\\
\textbf{Exercise \ref{exercise:groups:T_subgroup}:}\\
a.\\
(i) 1 is in $T$ because $|1|=1$\\
(ii) Let $c_1,c_2\in T\implies |c_1|=|c_2|=1$\\
$\implies |c_1c_2|=|c_1||c_2|=1$\\
$\implies c_1c_2\in T$\\
(iii) Let $c\in T \implies |c^{-1}|=\displaystyle\frac{1}{|c|}=1$\\
$\implies c^{-1}\in T$\\
So $T$ is a subgroup of $C^*$\\
\\
\textbf{Exercise \ref{exercise:groups:67}:}\\
a. $H$ is a subset of $T$\\
$H$ is a subgroup of $T$\\
b. $|H|=4$\\
c. $H$ is commutative\\
\\
\textbf{Exercise \ref{exercise:groups:69}:}\\
a. $Q^*$ is a subgroup of $R^*$ because:\\
(i) $1=\displaystyle\frac{1}{1}\in Q^*$\\
(ii) Let $\displaystyle\frac{p}{q},\frac{m}{n}\in Q^*\implies \frac{p}{q}.\frac{m}{n}=\frac{pm}{qn}\in Q^*$\\
(iii) $\displaystyle(\frac{p}{q})^{-1}=\frac{q}{p}\in Q^*$\\
(iv) $Q^*$ is a subset of $R^*$ (by algebra)\\
b. $Q^*$ is commutative\\
\\
\textbf{Exercise \ref{exercise:groups:73}:}\\
a. $Q^*$ is a subgroup of $R^*$ (Proposition 69)\\
\\
\textbf{Exercise \ref{exercise:groups:Z_construct_-1}:}\\
$\langle-1\rangle=\{...,-2,-1,0,1,-1,-2\}=\{n(-1):n\in Z\}=Z$\\
\\
\textbf{Exercise \ref{exercise:groups:Z6_construct_5}:}\\
$\langle 5\rangle=Z_6=\{0,1,2,3,4,5\}$\\
\\
\textbf{Exercise \ref{exercise:groups:82}:}\\
$G=\langle a\rangle$\\
$\implies G$ contains $a^{-1},e,a$\\
$\langle a^{-1}\rangle$ must contain $a^{-1},e,(a^{-1})^{-1}=a$\\
$\implies G=\langle a\rangle=\langle a^{-1}\rangle$\\
\\
\textbf{Exercise \ref{exercise:groups:U9_construct}:}\\
$U(9)=\{1,2,4,5,7,8\}$\\
$\langle 5\rangle=\{5^n\pmod{9}:n\in Z\}=U(9)$\\
\\
\textbf{Exercise \ref{exercise:groups:subgroup_3Z}:}\\
$4Z\neq\emptyset$ because $0,4,8,...\in 4Z$\\
Let $a,b\in Z\implies a=4m$ and $b=4n$\\
$\implies b^{-1}=-4n$\\
$\implies a+b^{-1}=4m-4n=4(m-n)\in 4Z$\\
$\implies 4Z$ is a subgroup of $Z$\\
\\
\textbf{Exercise \ref{exercise:groups:87}:}\\
a. $H=\{...\displaystyle\frac{1}{4},\frac{1}{2},1,2,4,...\}$\\
\\
b. Let $x\in H\implies x=\displaystyle\frac{2^n}{1}(n\in Z)\implies x\in Q^*$\\
$\implies H\subset Q^*$\\
c. Let $2^k,2^l\in H\implies 2^k2^l=2^{k+l}\in H$\\
$\implies H$ is closed under multiplication.\\
d. Let $2^k\in H\implies (2^k)^{-1}=2^{-k}\in H$\\
$\implies H$ is closed under inverse.\\
e. $H$ is a subgroup of $Q^*$ because $H$ has the identity element and is closed under multiplication and inverse.\\
\\
\textbf{Exercise \ref{exercise:groups:90}:}\\
$a\in G$ and $a\neq e$\\
$G=\{...,a^{-2},a^{-1},a,a^1,a^2,a^3,...,a^n\}$\\
Since $G$ is a finite group, elements can't all be different.\\
$\implies$  We can find $k,l\in Z$ and $k>l$ such that $a^k=a^l$\\
$\implies a^k(a^l)^{-1}=a^l(a^l)^{-1}$\\
$\implies a^{k-l}=e$\\
Let $m=k-l\in Z\to$ there is $m>0$ such that $a^m=e$\\
\\
\textbf{Exercise \ref{exercise:groups:Z6_orders}:}\\
a. $|1|=6$\\
$|5|=6$\\
b. $|e|=1$\\
\\
\textbf{Exercise \ref{exercise:groups:U9_orders}:}\\
$U(9)=\{1,2,4,5,7,8\}$\\
$|1|=1$\\
$|2|=6$ because $2^6\equiv 1\pmod{9}$\\
$|4|=3$ because $4^3\equiv 1\pmod{9}$\\
$|5|=6$ because $5^6\equiv 1\pmod{9}$\\
$|7|=3$ because $7^3\equiv 1\pmod{9}$\\
$|8|=2$ because $8^2\equiv 1\pmod{9}$\\
Cyclic subgroups:\\
$\langle1\rangle=\{1\}$\\
$\langle2\rangle=\{2,4,8,7,5,1\}=U(9)$\\
$\langle4\rangle=\{4,7,1\}$\\
$\langle5\rangle=\{5,7,8,4,2,1\}=U(9)$\\
$\langle7\rangle=\{7,4,1\}$\\
$\langle8\rangle=\{8,1\}$\\
\\
\textbf{Exercise \ref{exercise:groups:OrderEltCyclic}:}\\
$a^n=e\implies \langle a\rangle$ has n elements $\{a^1,a^2,a^3,...,a^n\}$\\
$\implies |a|=|\langle a\rangle|$\\
\\
\textbf{Exercise \ref{exercise:groups:97}:}\\
$|a|=n\implies a^n=e\implies a^na^{-1}=ea^{-1}\implies a^{n-1}=a^{-1}$\\
Let $m=n-1 \implies a^{-1}=a^m$\\

\subsection{Answers for additional group and subgroup exercises}
\textbf{Exercise \ref{ex:eoc:9}:}\\
Let $A=\begin{pmatrix}
x & y\\
u & v
\end{pmatrix}$ and $B=\begin{pmatrix}
m & n\\
o & p
\end{pmatrix}$\\
We have $\det(A)=xv-yu$ and $\det(B)=mp-on$\\
We also have $AB=\begin{pmatrix}
xm+yo & xn+yp\\
um+vn & un+vp
\end{pmatrix}$\\
So we have $\det(AB)=...=xmvp+youn-umyp-voxn=\det(A).\det(B)$ in $GL_2(R)$.\\
$A\in GL_2(R) \to$ A is invertible $\implies \det(A)\neq 0$\\
$B\in GL_2(R) \to$ B is invertible $\implies \det(B)\neq 0$\\
$\det(AB)=\det(A).\det(B)\implies \det(AB)\neq 0$\\
$\implies AB$ is invertible $\implies AB\in GL_2(R)$\\
So $GL_2(R)$ is closed under binary operation.\\
\\
\textbf{Exercise \ref{ex:eoc:10}:}\\
Let $x,y\in Z_2=\{0,1\}$\\
$\implies x\oplus y\in Z_2$ (because $Z_2$ closed under modular addition) (1)\\
Let $Z_2^n=\{(a_1,a_2,...,a_n):a_i\in Z_2\}$\\
Let $a=\{(a_1,a_2,...,a_n):a_i\in Z_2^n\}$ and $b=\{(b_1,b_2,...,b_n):b_i\in Z_2^n\}$\\
Then we have $a+b=(a_1,a_2,...,a_n)+(b_1,b_2,...,b_n)=(a_1+b_1,a_2+b_2,...,a_n+b_n)$\\
According to (1), we have $a_1+b_1\in Z_2$, $a_2+b_2\in Z_2$,...,$a_n+b_n\in Z_2$\\
$\implies a+b\in Z_2^n$\\
$\implies Z_2^n$ is closed under this operation.\\
(i) identity element is $(0,0,...,0)\in Z_2^n$\\
(ii) We have $(a_1,a_2,...,a_n)+(a_1,a_2,...,a_n)=(0,0,...,0)$\\
$\implies$  each element has its own inverse in $Z_2^n$\\
(iii) Associative: Let $a,b,c\in Z_2^n$, then we have:\\
$(a+b)+c =[(a_1,a_2,...,a_n)+(b_1,b_2,...,b_n)]+(c_1,c_2,...,c_n)$\\
$=(a_1+b_1,a_2+b_2,...,a_n+b_n)+(c_1,c_2,...,c_n)$\\
$=(a_1+b_1+c_1,a_2+b_2+c_2,...,a_n+b_n+c_n)$\\
$=(a_1,a_2,...,a_n)+(b_1+c_1,b_2+c_2,...,b_n+c_n)$\\
$=(a_1,a_2,...,a_n)+[(b_1,b_2,...,b_n)+(c_1,c_2,...,c_n)]$\\
$=a+(b+c)$\\
Therefore, $Z_2^n$ is a group under the defined operation.\\
\\
\textbf{Exercise \ref{ex:eoc:11}:}\\
Let $G$ is a group with 6 elements. We have:\\
$e\in G$\\
if $a\in G\implies a^{-1}\in G$\\
if $b\in G\implies b^{-1}\in G$\\
we need $ab\in G$ and $ba\in G$ to make G a group.\\
$\implies G$ must contains $a,b,e,a^{-1},b^{-1},ab,ba$\\
However, because $G$ has only 6 elements, we can make $ab=ba$. But then we also need $(ab)^{-1}\in G\implies G$ must have more than 6 elements.\\
Therefore, we disprove that every group containing 6 elements is abelian.\\
(Or you can give a specific counterexample).\\
\\
\textbf{Exercise \ref{ex:eoc:15}:}\\
$(g_1g_2...g_n)(g_n^{-1}g_{n-1}^{-1}...g_1^{-1}=g_1g_2...(g_ng_n^{-1})g_{n-1}^{-1}...g_1^{-1}$\\
$=g_1g_2...(g_{n-1}eg_{n-1}^{-1})...g_1^{-1}$\\
$=g_1g_2...(g_{n-1}g_{n-1}^{-1})...g_1^{-1}$\\
$=...$\\
$=g_1g_1^{-1}=e$\\
Similarly, we have $(g_n^{-1}g_{n-1}^{-1}...g_1^{-1})(g_1g_2...g_n)=e$\\
Therefore $g_n^{-1}g_{n-1}^{-1}...g_1^{-1}$ is the inverse of $g_1g_2...g_n$\\
\\
\textbf{Exercise \ref{ex:eoc:31}:}\\
Given that $H$ and $K$ are subgroups of $(G,\circ)$\\
(i) Let $x,y\in H\cap K \implies x,y\in H$ and $x,y\in K$\\
$\implies x\circ y\in H$ and $x\circ y\in K$ (because $H$ and $K$ are subgroups)\\
$\implies x\circ y\in H\cap K$\\
$\implies H\cap K$ is closed under $\circ$\\
(ii) $H$ and $K$ are subgroups $\implies e\in H$ and $e\in K$ by def. of subgroup\\
$\implies e\in H\cap K$\\ (by def. of intersection)\\
(iii) Let $x\in H\cap K \implies x\in H$ and $x\in K$ (def. of intersection)\\
$\implies x^{-1}\in H$ and $x^{-1}\in K$\\
$\implies x^{-1}\in H\cap K$\\
$\implies H\cap K$ is closed under inverse.\\
Therefore, the intersection of two subgroups of a group $G$ is also a subgroup of $G$.\\

\section{Solutions for ``Group Actions''}
\noindent\textbf{\textit{ (Chapter \ref{actions}})}\bigskip

\noindent\textbf{Exercise \ref{exercise:actions:Cube1}}(a)
$r_y$ or $r_y^{-3} $ takes  $z_-\rightarrow x_-$, $r_y\compose r_z^2$ or  $r_y^{-1}$ or $r_y^3$ takes $z_-\rightarrow x_+$, $r_x^3$ or $r_x^{-1}$ takes $z_-\rightarrow y_-$, $r_x$ or $r_x^{-3}$ takes $z_-\rightarrow y_+$,

\noindent\textbf{Exercise \ref{exercise:actions:Cube1}}(b)
$r_z^{-1}$or $r_x^{3}$ takes $y_-\rightarrow x_-$, $r_z$ or $r_z^{-3}$ takes $y_-\rightarrow x_+$, $r_z^{-2}$ or $r_z^{2}$ takes $y_-\rightarrow y_+$, $r_x$ or $r_x^{-3}$ takes $y_-\rightarrow z_-$, $r_x^{-1}$ or $r_x^{3}$ takes $y_-\rightarrow z_+$,

\noindent\textbf{Exercise \ref{exercise:actions:Cube1}}(c)
$r_x^2$ or $r_x^{-2}$ takes $+-\,-\rightarrow +++$,
$r_x$ or $r_x^{-3}$ takes $+-\,-\rightarrow -\,++$, 
$r_x\compose r_y^2$ takes $+-\,-\rightarrow -\,++$, 

\noindent\textbf{Exercise \ref{exercise:actions:CountingFormula2}}(a)
$G_{\overline{x_+,z_+}}=\{r_x^2\compose r_y, \var{id}\}$

\noindent\textbf{Exercise \ref{exercise:actions:CountingFormula2}}(b)
$G_{\overline{x_-,y_+}}=\{r_y^2\compose r_z^{-1}, \var{id}\}$

\noindent\textbf{Exercise \ref{exercise:actions:CountingFormula2}}(d)
 $G_{+,+,+}=\{\var{id}, r_y\compose r_z, r_z^{-1}\compose r_y^{-1} \}$

\noindent\textbf{Exercise \ref{exercise:actions:CountingFormula2}}(e)
$G_{+\,-\,+}=\{\var{id}, r_y\compose r_z^{-1}, r_z\compose r_y^{-1} \}$

\noindent\textbf{Exercise \ref{exercise:actions:Stabilizers1}}(a)
Two faces meet at each edge so beside the identity there is 1 rotation that leaves each edge fixed.  

\noindent\textbf{Exercise \ref{exercise:actions:Stabilizers1}}(b)
order 2 

\noindent\textbf{Exercise \ref{exercise:actions:Stabilizers1}}(c)
Edges come in pairs a rotation that stabilizes $\overline{x_+,z_+}$ will also stabilize $\overline{x_-,z_-}$ and $\overline{y_-,z_+}$ is stabilized by the same rotation as $\overline{y_+,z_-}$

\noindent\textbf{Exercise \ref{exercise:actions:Stabilizers1}}(d)
There are 6 pairs of edges each stabilized by one element of order two as well as the identity.  There are 6 group elements in the stabilizers of the different edges.

\noindent\textbf{Exercise \ref{exercise:actions:Stabilizers2}}

\begin{center}
\begin{tabular}{ |l |c|c| r |}\hline
  Number of elements & order of the element & Type of set that it stabilizes \\ \hline
  1 & 1 & entire cube (identity) \\ \hline
  6 & 4 & face \\ \hline
 3 & 2 & face \\ \hline
8 & 3& vertex \\ \hline
6& 2 & edge \\ \hline
\end{tabular}
\end{center}

\noindent\textbf{Exercise \ref{exercise:actions:Tetra 1}}
(a)  120 degrees~~(b)  order 3

\noindent\textbf{Exercise \ref{exercise:actions:Tetra4}}
$ \cal{O}_A =\{A,B,C,D\}$.

\noindent\textbf{Exercise \ref{exercise:actions:Tetra5}}
(a) $X_{r_{Bb}}=\{B,b\}$~~(b)~ $X_{ r_{Bb}\compose r_{Db }}=\{A,a\}$

\noindent\textbf{Exercise \ref{exercise:actions:Tetra9}}

\begin{tabular}{ l c r }
Number of elements & order of the element & Type of set that it stabilizes \\
1 & 1 & entire tetrahedron (identity) \\
 8 & 3 & face and vertex \\
 3 & 2 & edge \\
\end{tabular}

\noindent\textbf{Exercise \ref{exercise:actions:Octa1}}(b)  8 faces, 12 edges, 6 vertices

\noindent\textbf{Exercise \ref{exercise:actions:Octa2}} $|r_z|=3$.

\noindent\textbf{Exercise \ref{exercise:actions:Octa5}}(a)~ $X_{r_{y}\compose r_{z}}=\{x_+,x_-\}$~~(b)
$X_{r_{y}}^2\compose r_{x}=\{\overline{y_+z_+},\overline{y_-z_-}\}$.

\noindent\textbf{Exercise \ref{exercise:actions:Octa10}}

\begin{tabular}{ l c r }
Number of elements & order of the element & Type of set that it stabilizes \\
1 & 1 & entire octahedron (identity) \\
8 & 3 & face  \\
6 & 2 & edge \\
6 & 4  & vertices\\
 3 & 2 &  vertices\\
\end{tabular}

\noindent\textbf{Exercise \ref{exercise:actions:Dodeca2}}(a)  order 5

\noindent\textbf{Exercise \ref{exercise:actions:Dodeca2}}(b) 
$G_{f_1}={\var{id},r_{f_1},r_{f_1}^2, r_{f_1}^3, r_{f_1}^4}$, $G_{f_1}$ will also stabilize the opposite (parallel) face.

\noindent\textbf{Exercise \ref{exercise:actions:Dodeca2}}(c) 
 $|G_{f_1}|=5$

\noindent\textbf{Exercise \ref{exercise:actions:Dodeca2}}(c) 
25 elements 4 rotations for each of 6 pairs of opposite faces, plus the identity.

\noindent\textbf{Exercise \ref{exercise:actions:Dodeca2}}(d) 
$|{\cal O}_{f_1}|=12$ The orbit of $f_1$ includes all faces of the dodecahedron.

\noindent\textbf{Exercise \ref{exercise:actions:Dodeca3}}
$5\times 12=60$

\noindent\textbf{Exercise \ref{exercise:actions:Dodeca4}}(a) 
order 3

\noindent\textbf{Exercise \ref{exercise:actions:Dodeca4}}(b) 
$G_{v_1}={\var{id},r_{v_1},r_{v_1}^2}$ $G_{v_1}$ will also stabilize the opposite vertex.

\noindent\textbf{Exercise \ref{exercise:actions:Dodeca4}}(c) 
 20 rotations 10 pairs of vertices, 2 rotations besides the identity stabilize each pair.

\noindent\textbf{Exercise \ref{exercise:actions:Dodeca4}}(d) 
$|{\cal O}_{v_1}|=20$ The orbit of $v_1$ includes all the vertices of the dodecahedron.

\noindent\textbf{Exercise \ref{exercise:actions:Dodeca5}}(a) ~2~~(b)~
There are 60 group elements in $G$.  By previous exercise 25 elements stabilize faces, 20 rotations stabilize vertices. We need 15 more elements to make up $G$.  There are $30/2=15$ pairs of edges, so edges must come in pairs.

\noindent\textbf{Exercise \ref{exercise:actions:Dodeca6}} (a)~
 15, one rotation for each pair of edges.

\noindent\textbf{Exercise \ref{exercise:actions:Dodeca6}} (b)~

\begin{tabular}{ l c r }
Number of elements & order of the element & Type of set that it stabilizes \\
1 & 1 & entire dodecahedron (identity) \\
24 & 5 & face  \\
15 & 2 & edge \\
20 & 3  & vertices\\
\end{tabular}

\noindent\textbf{Exercise \ref{exercise:actions:Eulers1}} (b)
 $|G|=2\cdot |{\cal O}_e|$

\section{Solutions for ``Polynomials''}
\noindent\textbf{\textit{ (Chapter \ref{poly})}}\bigskip

\noindent\textbf{Exercise \ref{exercise:poly:poly1}}
\begin {enumerate} [(a)]
\item
 Sum: $x^3$ , Product: $x^5+x^3+x^2+1$
\item
 Sum: $x^3+1$ , Product: $x^8+x^7+x^5+x^3+x^2$
\item
 Sum: $0$ , Product: $x^8+x^6+x^4+x^2+1$
\end {enumerate}

\noindent\textbf{Exercise \ref{exercise:poly:poly2}}
\begin {enumerate} [(a)]
\item
 $\mathbb{Z}[x]$,$\mathbb{Q}[x]$,$\mathbb{R}[x]$,$\mathbb{C}[x]$,$\mathbb{Z}_5[x]$,$\mathbb{Z}_6[x]$
\item
 None
\item
 $\mathbb{Q}^*[x],\mathbb{R}^*[x], \mathbb{C}^*[x], {\mathbb{Z}_5}^*[x]$
\item
 None
\end {enumerate}

\noindent\textbf{Exercise \ref{exercise:poly:poly3}}
\begin {enumerate} [(a)]
\item
 Sum: $x^3+x+1$ , Product: $2x^5+2x^4+4x^3+3x^2$
\item
 Sum: $2x^4+4x^3$ , Product: $2x^7+x^6+4x^5+4x^3+4x^2+5$
\end {enumerate}

\noindent\textbf{Exercise \ref{exercise:poly:poly4}}
\begin {enumerate} [(a)]
\item
 $\sum_{i=0}^{4} (2i+1)x^2i$, degree: 8
\item
 $\sum_{i=0}^{3} \cis(i\pi/2)x^2i$, degree: 6
\item 
$\sum_{i=0}^{3} (7i^3-18i^2+13i+1)x^i$, degree: 3
\item
 $\sum_{i=0}^{5} ((-1)^i/(2i+1))x^i$, degree: 5
\end {enumerate}

\noindent\textbf{Exercise \ref{exercise:poly:poly5}}
\begin {enumerate} [(a)]
\item
 No
\item 
Yes
\item
 n is a prime number.
\end {enumerate}

\noindent\textbf{Exercise \ref{exercise:poly:poly6}}
\begin {enumerate} [(a)]
\item
 $p(x)=2x$, $q(x)=2x^3+2$
\item 
No
\item
 $p(x) \cdot q(x)=0$, where $p(x)=3x^3$ and $q(x)=2x^2$ are in $\mathbb{Z}_6[x]$.
\end {enumerate}

\noindent\textbf{Exercise \ref{exercise:poly:mult2way}}
\begin {enumerate} [(a)]
\item 
 $x^3-2x^2-15x$
\item
 $5x^4-5\sqrt{3}x^2+2\sqrt{3}x-6$
\item
$4x^5-3x^4+\frac{7}{2}x^3+8x^2-6x+7$
\item
$80x^7-40x^6+64x^5-90x^4+47x^3-21x^2$
\end {enumerate}

\noindent\textbf{Exercise \ref{exercise:poly:multform}}
\begin {enumerate} [(a)]
\item 
 $x^6+2x^5+3x^4+4x^3+3x^2+2x+1$
\item
 $2x^5+5x^4+8x^3+3x^2$
\item
$2x^8+x^7-2x^6-6x^5-10x^4-2x^3+4x^2+7x+6$
\item
$12x^8+6x^7-12x^6-36x^5-60x^4-12x^3+24x^2+42x+36$
\end {enumerate}

\noindent\textbf{Exercise \ref{exercise:poly:div1}}
\begin {enumerate} [(a)]
\item 
 $q(x)=x+5$, $r(x)=37$
\item
 $q(x)=15x^2+75x+388$, $r(x)=1913$
\item
$q(x)=5x-\frac{1}{2}$, $r(x)=23x+25$
\end {enumerate}

\noindent\textbf{Exercise \ref{exercise:poly:div1}}
\begin {enumerate} [(a)]
\item
$3x^5 + 2x^4 + 3x^3 + x^2 +2x +4$
\item
$x^6 + x^5 + x^2 + x$
\end {enumerate}
%\end {answer}



%
%\backmatter
%% bibliography, glossary and index would go here.
%
%\end{document}