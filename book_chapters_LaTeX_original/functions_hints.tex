\section{Hints for ``Functions: basic concepts'' exercises}\label{sec:functions:hints}

\noindent Exercise \ref{exercise:functions:funtable}(e): There is a formula of the form $f(x) = ax^2 + bx + c$

\noindent Exercise \ref{exercise:functions:45}: Can there be any elements in the codomain that are not in the range?.

\noindent Exercise \ref{exercise:functions:NxNBijection}(a): Consider the values $f(1,i)$  for $i=1,2,3,\ldots$. (b): Consider  the values $f(2,j)$ and $f(1,i)$.

\noindent Exercise \ref{exercise:functions:NxN}(a): Given any element $(i,j)$ of $\mathbb{Z} \times \mathbb{Z}$, set $i=m+n$ and $j=m+2n$ and solve for $m$ and $n$ in terms of $i$ and $j$.

\noindent Exercise \ref{exercise:functions:NxN}(b): Suppose that $g(m,n) = g(p,q)$. It follows that $(m + n, m + 2n) = (p + q, p + 2q)$. %This gives two separate equations:  $m+n=p+q$ and $m+2n = p+2q$.

\noindent Exercise \ref{exercise:functions:BijectionComposeExer}(a): You just need to show that $g \compose f$ is both one-to-one and onto. Use the previous exercises.

\noindent Exercise \ref{InverseUniqueExers-unique}: Prove by contradiction. Suppose that there exists a $y$ such that $g_1(y) \neq g_2(y)$.  Then what can you say about $f \compose g_1(y)$ and  $f \compose  g_2(y)$?
